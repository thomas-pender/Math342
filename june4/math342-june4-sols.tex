\documentclass[a4paper,11pt]{article}

\usepackage[utf8]{inputenc}
\usepackage[english]{babel}
\usepackage{amssymb, amsmath, amsthm, mathrsfs}
\usepackage[left=1.0in,right=1.0in,top=1.0in,bottom=1.0in]{geometry}
\usepackage[T1]{fontenc}
\usepackage{array}
\usepackage{longtable}
\usepackage{multirow}
\usepackage{calc}
\usepackage[inline,shortlabels]{enumitem}
\usepackage{changepage}
\usepackage{booktabs}
\usepackage{capt-of}
\usepackage{subcaption}
\usepackage[leftcaption]{sidecap}
\usepackage[numbers]{natbib}
\usepackage{times}
\usepackage{titlesec}
\usepackage{xcolor}
\usepackage{lineno}
\usepackage{xpatch}
\xpatchcmd\swappedhead{~}{.~}{}{}
\allowdisplaybreaks

\newtheoremstyle{mythm}
{}                % Space above
{}                % Space below
{\itshape}        % Theorem body font % (default is "\upshape")
{1.5em}                % Indent amount
{\scshape}       % Theorem head font % (default is \mdseries)
{.}               % Punctuation after theorem head % default: no punctuation
{0.5em}               % Space after theorem head
{}                % Theorem head spec
\theoremstyle{mythm}


\newtheorem*{theorem*}{Theorem}
\newtheorem{theorem}{Theorem}
\newtheorem{fact}[theorem]{Fact}
\newtheorem{proposition}[theorem]{Proposition}
\newtheorem{lemma}[theorem]{Lemma}
\newtheorem{corollary}[theorem]{Corollary}
\newtheorem{question}[theorem]{Question}
\newtheorem{result}[theorem]{Result}
\newtheorem{observation}[theorem]{Observation}
\newtheorem{conjecture}[theorem]{Conjecture}

\newtheoremstyle{mydef}
{}                % Space above
{}                % Space below
{}        % Theorem body font % (default is "\upshape")
{1.5em}                % Indent amount
{\scshape}       % Theorem head font % (default is \mdseries)
{.}               % Punctuation after theorem head % default: no punctuation
{0.5em}               % Space after theorem head
{}                % Theorem head spec
\theoremstyle{mydef}

\newtheorem{example}[theorem]{Example}
\newtheorem{definition}[theorem]{Definition}
\newtheorem{remark}[theorem]{Remark}
\newtheorem*{remark*}{Remark}

\makeatletter
\renewenvironment{proof}[1][\proofname]{\par
  \pushQED{\qed}%
  \normalfont \topsep6\p@\@plus6\p@\relax
  \trivlist
\item\relax
  {\hspace{1.5em}\itshape
    #1\@addpunct{.}}\hspace\labelsep\ignorespaces
}{%
  \popQED\endtrivlist\@endpefalse
}
\makeatother

\def\Box{\hskip1ex\vbox{\hrule height0.6pt\hbox{%
      \vrule height1.3ex width0.6pt\hskip0.8ex
      \vrule width0.6pt}\hrule height0.6pt
  }}
\renewcommand{\qed}{\Box}

\newcommand{\red}[1]{\textcolor{red}{#1}}
\newcommand{\blue}[1]{\textcolor{blue}{#1}}
\newcommand{\purple}[1]{\textcolor{magenta}{#1}}
\newcommand{\ddet}{\text{det}}
\renewcommand{\pmod}[1]{\text{ (mod $#1$)}}
\newcommand{\mmod}[2]{#1\text{ mod }#2}
\newcommand{\abs}[1]{\left\vert #1 \right\vert}
\newcommand{\C}{\mathbf{C}}
\newcommand{\Z}{\mathbf{Z}}
\newcommand{\N}{\mathbf{N}}
\newcommand{\LL}{\mathscr{G}}
\newcommand{\z}{\mathbin{\ooalign{$\hidewidth i \hidewidth$\cr$\phantom{+}$}}}
\newcommand{\y}{\mathbin{\ooalign{$\hidewidth j \hidewidth$\cr$\phantom{+}$}}}
\newcommand{\gf}{\text{GF}}

\newcolumntype{R}{>{\scriptsize}r}
\newcolumntype{L}{>{\scriptsize}l}
\newcolumntype{C}{>{\scriptsize}c}

\renewcommand{\citenumfont}[1]{\textbf{#1}}
\renewcommand{\bibnumfmt}[1]{\textbf{#1.}}

\titleformat{\section}{\normalfont\Large\bfseries\centering}{\thesection.}{0.5em}{}
\titleformat{\subsection}{\normalfont\bfseries}{\thesubsection.}{0.5em}{}

\newenvironment{myabstract}{\vspace{1em}\begin{adjustwidth}{3em}{3em}\begin{small}\textbf{Abstract.}}{\end{small}\end{adjustwidth}\vspace{1em}}
\newenvironment{mykeywords}{\vspace{1em}\begin{adjustwidth}{3em}{3em}\begin{small}\textbf{Keywords.}}{\end{small}\end{adjustwidth}\vspace{1em}}

\DeclareCaptionLabelSeparator{custom}{.}
\DeclareCaptionLabelFormat{custom}
{%
  \textsc{#1 #2}
}
\DeclareCaptionFormat{custom}
{%
  #1#2 #3
}
\captionsetup
{
  format=custom,%
  labelformat=custom,%
  labelsep=custom
}

\begin{document}

\begin{center}
  {\Large\bfseries Math 342 Tutorial} \\
  {\normalsize\bf June 4, 2025}
\end{center}

\noindent{\bf Question 1.}
\begin{enumerate*}[{\bf (a)}]
\item What is the remainder of when $5^{100}$ is divided by $7$?
\item What is the remainder when $18!$ is divided by $437$?
\end{enumerate*}

\blue{
  \begin{enumerate}[{\bf (a)}]
  \item $5^{100} = 5^{(16)(6)}5^4 \equiv 5^4 \equiv 2 \pmod{7}$.
  \item We have $437=23 \cdot 19$. By Wilson's Theorem
    \[
      -1 \equiv 22! \equiv 22 \cdot 21 \cdot 20 \cdot 19 \cdot 18! \equiv
      (-1)(-2)(-3)(-4)18! \equiv 18! \pmod{23}.
    \]
    and
    \[
      -1 \equiv 18! \pmod{19}.
    \]
    But then $-1 \equiv 18! \pmod{19 \cdot 23}$, i.e. $-1 \equiv 18!
    \pmod{437}$.
  \end{enumerate}
}

\noindent{\bf Question 2.} Show that if $p$ is an odd prime, then $2(p-3) \equiv
-1 \pmod{p}$. \\

\blue{By Wilson's Theorem,
  \[
    -1 \equiv (p-1)! \equiv (p-1)(p-2)(p-3)! \equiv (-1)(-2)(p-3)! \equiv
    2(p-3)! \pmod{p}.
  \]
}

\noindent{\bf Question 3.} Show that if $n$ is a composite integer with $n \neq
4$, then $(n-1)! \equiv 0 \pmod{n}$. \\

\blue{If $n$ is composite, then $n=ab$ with $1 < a \leqq b < n$. If the only
  proper factorzation of $n$ is as $n=a^2$, then $a$ must be a prime number.}

\blue{For the first case, assume $n=ab$ with $a \neq b$. Then both $a,b$ divide
  $(n-1)!$ by definition, hence also $n \mid (n-1)!$. If $n=p^2$ for some prime
  $p$, then since $n > 4$ in this case, it is also true that $2p < n$ (why?).
  But then $p,\,2p \mid (n-1)!$, hence $2n \mid (n-1)!$ so that $n \mid (n-1)!$
  as $n \mid 2n$.} \\

\noindent{\bf Question 4.} Show that $1^{p-1}+2^{p-1}+\cdots+(p-1)^{p-1} \equiv
-1 \pmod{p}$. \\

\blue{
  For $k$ such that $1 \leqq k < p$, we have that $k^{p-1} \equiv 1 \pmod{p}$.
  So,
  \[
    1^{p-1}+2^{p-1}+\cdots+(p-1)^{p-1} \equiv \underbrace{1+1+\cdots+1}_{p-1}
    \equiv p-1 \equiv -1 \pmod{p}.
  \]
}

\noindent{\bf Question 5.}
\begin{enumerate*}[{\bf (a)}]
\item Let $p$ be a prime, and let $n_1,\dots,\,n_k$ ($k>1$) be integers between
  0 and $p$ inclusive such that $n_1+\cdots+n_k=p$ Show that
  $\binom{p}{n_1,\dots,n_k} \equiv 0 \pmod{p}$ whenever each $n_i < p$.
\item Show that $(x_1+\cdots+x_n)^{p^e} \equiv x_1^{p^e}+\cdots+x_n^{p^e} \pmod{p}$.
\item Show that $1^p+2^p+\cdots+(p-1)^p \equiv 0 \pmod{p}$ in two ways.
\end{enumerate*}

\blue{
  \begin{enumerate}[{\bf (a)}]
  \item The proof is by induction on $k$. Let $k=2$. Then
    $\binom{p}{n_1,n_2}=\binom{p}{n_1}$ since $n_2=p-n_1$. The assumption that
    each $n_1,\,n_2 < p$ means in particular that $0 < n_1 < p$. For simplicity,
    write $n=n_1$. Then
    \[
      \binom{p}{n} = \frac{p(p-1)\cdots(p-n+1)}{n!}.
    \]
    Since $(p,\,n!)=1$, we must have that
    \[
      \frac{(p-1)\cdots(p-n+1)}{n!}
    \]
    is an integer. Hence, $\binom{p}{n} \equiv 0 \pmod{p}$. This shows the base
    case. \\
    Next, assume that $k \geqq 2$ and the result holds for this given $k$. Let
    $n_1,\dots,n_{k+1}$ be nonnegative integers such that $\sum_{i=1}^kn_i=p$
    and each $n_i < p$. If $n_{k+1}=0$, then
    \[
      \binom{p}{n_1,\dots,\,n_{k+1}} = \binom{p}{n_1,\dots,\,n_k} \equiv 0\pmod{p}
    \]
    by the inductive hypothesis. If $n_{k+1} > 0$, then
    \[
      \binom{p}{n_1,\dots,\,n_{k+1}}=\binom{p}{n_{k+1}}\binom{p-n_{k+1}}{n_1,\dots,n_k}
      \equiv 0\pmod{p}
    \]
    by the base case. The result now follows by mathematical induction.
  \item The proof is by double induction. The base case $e=0$ and $n=1$ is
    trivially true. Also trivial is the first inductive step that if $e=0$, and
    if the result holds for a given $n \geqq 1$ with $e=0$, it also holds for
    $n+1$. Next, let $e \geqq 0$ be given, and assume the result holds for all
    $n \geqq 1$ with this given $e$. Then
    \begin{align*}
      (x_1 + \cdots x_n)^{p^{e+1}} &\equiv ((x_1+\cdots+x_n)^{p^e})^p \\
      &\equiv (x_1^{p^e}+\cdots+x_n^{p^e})^p \\
      &\equiv \sum_{\begin{smallmatrix}n_1,\dots,n_k \\ n_1+\cdots+n_k=p\end{smallmatrix}}\binom{p}{n_1,\dots,n_k} (x_1^{n_1}\cdots x_k^{n_k})^{p^e} \\
      &\equiv x_1^{p^{e+1}}+\cdots+x_n^{p^{e+1}} \pmod{p}
    \end{align*}
    by the inductive hypothesis and part (a). The result now follows from
    induction.
  \item If $p$ is even the result is trivial. So assume $p$ is odd. By Fermat's
    Little Theorem, $1^p+2^p+\cdots+(p-1)^p\equiv \frac{p(p-1)}{2} \pmod{p}$.
    Since $p$ is odd, $\frac{p-1}{2}$ is an integer, hence the sum is divisible
    by $p$. \\
    By part (b) instead, we have that $1^p+2^p+\cdots+(p-1)^p \equiv
    (1+2+\cdots+(p-1))^p \equiv (\frac{p(p-1)}{2})^p \equiv 0^p \equiv 0
    \pmod{p}$.
  \end{enumerate}
}

\noindent{\bf Question 6.}
\begin{enumerate*}[{\bf (a)}]
\item Show that if $p$ and $q$ are distinct primes, then $p^{q-1}+q^{p-1} \equiv
  1 \pmod{pq}$.
\item Show that if $p$ is prime, then $p \mid (a^p+(p-1)!a)$.
\item Show that if $p$ is an odd prime, then $1^23^2 \cdots (p-4)^2(p-2)^2$ is
  equivalent to $(-1)^{(p+1)/2}$ modulo $p$.
\end{enumerate*}

\blue{
  \begin{enumerate}[{\bf (a)}]
  \item We have that $(p,\,q)=1$. By Fermat's Little Theorem,
    \[
      1 \equiv p^{q-1} \equiv p^{q-1} + q^{p-1} \pmod{q}.
    \]
    Similarly,
    \[
      1 \equiv q^{p-1} \equiv q^{p-1} + p^{q-1} \pmod{p}.
    \]
    Hence,
    \[
      p^{q-1} + q^{p-1} \pmod{pq}.
    \]
  \item Observe, $a^p+(p-1)!a \equiv a(1+(p-1)!) \equiv 0 \pmod{p}$. The first
    congruence is by Fermat's Little Theorem; the second by Wilson's Theorem.
  \item We have
    \begin{align*}
      1^23^3 \cdots (p-4)^2(p-2)^2&=(1\cdot1)(3\cdot3)\cdots(p-4)(p-4)(p-2)(p-2)\\
      &= (-1)^{(p-1)/2}(1)(-1)(3)(-3) \cdots (p-4)(4-p)(p-2)(2-p) \\
      &= (-1)^{(p-1)/2}(1)(p-1)(3)(p-3) \cdots (p-4)4(p-2)2.
    \end{align*}
    Since the odd positive integers less than $p$ are $1,\,3,\dots,\,p-4,\,p-2$,
    and the even positive integers less than $p$ are $2,\,4,\dots,\,p-3,\,p-1$,
    we have that
    \begin{align*}
      1^23^3 \cdots (p-4)^2(p-2)^2 &= (-1)^{(p-1)/2}(p-1)! \\
      \equiv (-1)^{(p-1)/2}(-1) \\
      &\equiv (-1)^{(p+1)/2} \pmod{p}
    \end{align*}
    by Wilson's Theorem.
  \end{enumerate}
}

The questions 7 and 9 each show by elementary means that $\phi$ is a
multiplicative arithmetical function (your textbook contains a third). Next
week, we will derive an algebraic proof using the Chinese Remainder Theorem.
Many of the results you are learning via elementary means admit more natural
interpretations in more general algebraic settings. We will see that the main
properties of the Euler Totient Function are easier understood in this more
general setting. \\

\noindent{\bf Question 7.}
\begin{enumerate*}[{\bf (a)}]
\item Suppose that $(m,\,m')=1$, and that $a$ runs through a complete system of
  residues modulo $m$, and that $a'$ runs through a complete system of integers
  modulo $m'$. Show that $a'm+am'$ runs through a complete set of residues
  modulo $mm'$.
\item Show that $\phi$ is multiplicative, that is, if $(m,\,m')=1$, then
  $\phi(mm')=\phi(m)\phi(m')$.
\end{enumerate*}

\blue{
  \begin{enumerate}[{\bf (a)}]
  \item There are $mm'$ numbers $a'm+am'$. If
    \[
      a_1'm+a_1m' \equiv a_2'm+a_2m' \pmod{mm'}
    \]
    then $a_1m' \equiv a_2m' \pmod{m}$, hence $a_1 \equiv a_2 \pmod{m}$.
    Similarly, $a_1' \equiv a_2' \pmod{m'}$. Thus, the $mm'$ such numbers are
    all incongruent and form a complete set of residues mod $mm'$.
  \item Let $a,\,a',\,m,\,m'$ be as in part (a). If $(a'm+am',\,mm')=1$, then
    $(a'm+am',\,m)=1=(a'm+am',\,m')$ which in turn implies that
    $(am',\,m)=1=(a'm,\,m)$ so that $(a,\,m)=1=(a',\,m')$ since $(m,\,m')=1$. We
    can also transpose this argument. Thus, the $\phi(mm')$ integers of the form
    $a'm+am'$ coprime to $mm'$ are in bijective correspondence with the
    $\phi(m)\phi(m')$ ordered pairs $(a,\,a')$ of integers with $(a,\,m)=1$ and
    $(a',\,m')=1$. Thus, $\phi(mm')=\phi(m)\phi(m')$, as desired.
  \end{enumerate}
}

Given two arithmetical functions $f$ and $g$, we define their Dirichlet product
$f*g$ as $(f*g)(n)=\sum_{d \mid n}f(d)g(\frac{n}{d})$. The function $\iota$ is
defined by $\iota(n)=\lfloor \frac{1}{n} \rfloor$. We call $\iota$ the
arithmetic identity. \\

\noindent{\bf Question 8.} Let $f$ and $g$ be two arithmetical functions. Show
the following.
\begin{enumerate*}[{\bf (a)}]
\item $f*g = g*f$.
\item $(f*g)*h=f*(g*h)$.
\item $f*\iota = \iota*f=f$.
\item $g$ is the arithmetic inverse of $f$ if $f*g=\iota$. Show the arithmetic
  function $f$ has an inverse if and only if $f(1) \neq 0$. Show that if $f$ has
  an arithmetic inverse, then it is unique.
\item If $h$ is the arithmetic inverse of $g$, then $(f*g)*h=f$.
\item If $f$ and $g$ are multiplicative, then so is $f*g$.
\end{enumerate*}

\blue{
  \begin{enumerate}[{\bf (a)}]
  \item Let $n$ be given. As $d$ ranges over the divisors $n/d$ ranges over the
    conjugate pairs. Hence,
    \[
      (f*g)(n) = \sum_{d \mid n}f(d)g\left( \frac{n}{d} \right)
      = \sum_{d \mid n}f\left( \frac{n}{d} \right)g(d) = (g*f)(n).
    \]
  \item We have
    \begin{align*}
      ((f*g)*h)(n) &= \sum_{d \mid n}(f*g)(d)h\left( \frac{n}{d} \right) \\
      &= \sum_{d \mid n}\left( \sum_{d' \mid d}f(d')g\left( \frac{d}{d'} \right) \right)h\left( \frac{n}{d} \right) \\
      &= \sum_{\begin{smallmatrix}d \mid n \\ d' \mid d\end{smallmatrix}}f(d')g\left( \frac{d}{d'} \right)h\left( \frac{n}{d} \right).
    \end{align*}
    Note that we are just summing over all triples $a,\,b,\,c$ such that
    $abc=n$. Therefore, we can write this as
    \begin{align*}
      \sum_{\begin{smallmatrix}d \mid n \\ d' \mid d\end{smallmatrix}}f(d')g\left( \frac{d}{d'} \right)h\left( \frac{n}{d} \right) &= \sum_{\begin{smallmatrix}d \mid n \\ d' \mid d\end{smallmatrix}}g(d')h\left( \frac{d}{d'} \right)f\left( \frac{n}{d} \right) \\
      &= \sum_{d \mid n}\left( \sum_{d' \mid d} g(d')h\left( \frac{d}{d'} \right) \right)f\left( \frac{n}{d} \right) \\
      &= \sum_{d \mid n}(g*h)(d)f\left( \frac{n}{d} \right) \\
      &= ((g*h)*f)(n) \\
      &= (f*(g*h))(n).
    \end{align*}
  \item Observe
    \[
      (f*\iota)(n)=\sum_{d \mid n}\iota(d)f\left( \frac{n}{d} \right) =
      \sum_{d \mid n}\left\lfloor \frac{1}{d} \right\rfloor f\left( \frac{n}{d}
      \right) = f(n).
    \]
  \item Note that $(f*g)(1)=\iota(1)$ reduces to $f(1)g(1)=1$. So if $f(1)=0$,
    then $f$ admits no arithmetic inverse. If $f(1) \neq 0$, then $g(1)$ is
    uniquely determined as $g(1)=1/f(1)$. Next, assume we have determined the
    values $g(k)$ for $k<n$ in such a way that $(f*g)(k)=\iota(k)$. We need to
    solve $(f*g)(n)=\iota(n)$, i.e.,
    \[
      0 = \sum_{d \mid n}g(d)f\left( \frac{n}{d} \right) =
      f(1)g(n) + \sum_{\begin{smallmatrix}d \mid n \\ d <
        n\end{smallmatrix}}g(d)f\left( \frac{n}{d} \right)
    \]
    But then $g(n)$ is uniquely determined as
    \[
      g(n) = \frac{-1}{f(1)}\sum_{\begin{smallmatrix}d \mid n \\ d <
        n\end{smallmatrix}}g(d)f\left( \frac{n}{d} \right)
    \]
    This shows at once existence and uniqueness.
  \item $(f*g)*h = f*(g*h) = f*\iota = f$.
  \item Suppose that $(m,\,n)=1$. There the divisors $d$ od $mn$ are in
    bijective correspondence with pairs $(a,\,b)$ such that $a \mid m$ and $b
    \mid n$; in particular, $d$ factors uniquely as $d=ab$ with $a \mid m$ and
    $b \min n$. Then further $(a,\,b)=1$ and $(m/a,n/b)=1$. \\
    Let $h=f*g$ where $f$ and $g$ are multiplicative. Then
    \begin{align*}
      h(mn) &= \sum_{d \mid mn}f(d)g\left( \frac{mn}{d} \right) \\
      &= \sum_{\begin{smallmatrix}a \mid m \\ b \mid n\end{smallmatrix}}f(ab)g\left( \frac{mn}{ab} \right) \\
      &= \sum_{\begin{smallmatrix}a \mid m \\ b \mid n\end{smallmatrix}}f(a)f(b)g\left( \frac{m}{a} \right)g\left( \frac{n}{b} \right) \\
      &= \left( \sum_{a \mid m}f(a)g\left( \frac{m}{a} \right) \right)\left( \sum_{b \mid n}f(b)g\left( \frac{n}{b} \right) \right) \\
      &= h(n)h(m),
    \end{align*}
    as required.
  \end{enumerate}
}

The M\"{o}bius function $\mu(n)$ is defined as follows: $\mu(1)=1$; if
$n=p_1^{e_1} \cdots p^{e_k}$, then $\mu(n)=(-1)^k$ if $e_1=\cdots=e_k=1$;
$\mu(n)=0$ otherwise. \\

\noindent{\bf Question 9.} Show the following.
\begin{enumerate*}[{\bf (a)}]
\item $\mu$ is multiplicative.
\item $\sum_{d \mid n}\mu(d)=\lfloor \frac{1}{n} \rfloor$.
\item Show that $\phi$ is multiplicative using part (b). [Hint: Argue that you
  can write $\phi(n)=\sum_{k = 1}^n\lfloor \frac{1}{(n,\,k)} \rfloor$]
\end{enumerate*}

\blue{
  \begin{enumerate}[{\bf (a)}]
  \item If $(m,\,n)=1$, then $mn$ has square factors if and only if $m$ or $n$
    does. If $n$ has square factors, then $\mu(n)=0$ so that
    $\mu(mn)=0=0\mu(m)=\mu(n)\mu(m)$. Similarly, if $m$ has square factors, then
    again $\mu(mn)=\mu(m)\mu(n)=0$. If $m=n=1$, then
    $\mu(mn)=\mu(1)=1=1\cdot1=\mu(1)\mu(1)=\mu(n)\mu)m$. If $m=1$ and
    $n=p_1 \cdots p_k$, then $\mu(mn)=\mu(n)=1\mu(n)=\mu(m)\mu)n$.
    Similarly, if $n=1$ and $m=q_1\cdots q_\ell$, then $\mu(mn)=\mu(m)\mu(n)$.
    If $n=p_1 \cdots p_k$ and $m=q_1 \cdots q_\ell$, then $\mu(mn)=\mu(p_1
    \cdots p_kq_1 \cdots q_\ell)=(-1)^{k+\ell}=(-1)^k(-1)^\ell=\mu(n)\mu(m)$.
  \item When $n=1$, we have $\mu(1)=1=\lfloor \frac{1}{1} \rfloor$. Let $n>1$
    and write $n=p_1^{e_1} \cdots p_k^{e_k}$. In the sum $\sum_{d \mid n}\mu(d)$,
    the only nonzero terms are $d=1$ and $d$ a product of distinct primes. Thus,
    \begin{align*}
      \sum_{d \mid n}\mu(d) &= 1+\sum_{p \mid n}\mu(p)+\sum_{\begin{smallmatrix}p_i,p_j \mid n \\ i < j\end{smallmatrix}}\mu(p_1p_2) + \cdots \\
      &= 1 + \binom{k}{1}(-1) + \binom{k}{2}(-1)^2 + \cdots + \binom{k}{k}(-1)^k=(1-1)^k = 0.
    \end{align*}
    This shows that indeed $\sum_{d \mid n}\mu(d) = \lfloor \frac{1}{n} \rfloor$.
  \item Observe that
    \[
      \phi(n) = \sum_{\begin{smallmatrix}k=1 \\ (k,\,n)=1\end{smallmatrix}}^n 1
      = \sum_{k=1}^n \left\lfloor \frac{1}{(k,\,n)} \right\rfloor.
    \]
    From part (b), we then have that
    \begin{align*}
      \phi(n) &= \sum_{k=1}^n \left\lfloor \frac{1}{(k,\,n)} \right\rfloor
      = \sum_{k=1}^n\sum_{d \mid (k,\,n)}\mu(d)
      = \sum_{k=1}^n\sum_{\begin{smallmatrix}d \mid k \\ d \mid n\end{smallmatrix}}\mu(d) \\
      &= \sum_{d \mid n}\sum_{q=1}^{n/d}\mu(d)
      = \sum_{d \mid n}\mu(d)\frac{n}{d}
      = (\mu * \text{id})(n)
    \end{align*}
    where $id(m)=m$ for each $m \in \N$. Since $\text{id}$ is trivially
    multiplicative, and since $\mu$ is multiplicative by part (a), we have that
    $\phi = \mu * \text{id}$ is multiplicative by Question 8(f).
  \end{enumerate}
}

\noindent{\bf Question 10.} Show the following properties of $\phi$.
\begin{enumerate}[{\bf (a)}]
\item $\phi(p^e)=p^e-p^{e-1}$ for every prime $p$ and $e>0$.
\item If $n=p_1^{e_1} \cdots p_k^{e_k}$, then $\phi(n)=n\prod_{p \mid
    n}(1-\frac{1}{p})$.
\item $\phi(mn)=\phi(m)\phi(n)d/\phi(d)$ where $d=(n,\,m)$.
\item $a \mid b$ implies $\phi(a) \mid \phi(b)$.
\item $\phi(n)$ is even for $n \geqq 3$. Moreover, if $n$ has $r$ distinct odd
  prime factors, then $2^r \mid \phi(n)$.
\end{enumerate}

\blue{
  \begin{enumerate}[{\bf (a)}]
  \item The only divisors of $p^e$ are $1,\,p,\,p^2,\dots,\,p^e$. The integers
    at most $p^e$ coprime to $p^e$ are then $2,\dots,p-1,p-2,\dots,p^2-1,\dots$.
    There are $p^{e-1}(p-1)=p^e-p^{e-1}$, i.e., $\phi(p^e)=p^e-p^{e-1}$.
  \item Observe
    \begin{align*}
      \phi(n) &= \phi(p_1^{e_1} \cdots p_k^{e^k}) = (p_1^{e_1}-p_1^{e_1-1}) \cdots (p_k^{e_k}-p_k^{e_k-1}) \\
      &= p^{e_1} \cdots p_k^{e_k}\left( 1-\frac{1}{p_1} \right) \cdots \left( 1-\frac{1}{p_k} \right) \\
      &= n\left( 1-\frac{1}{p_1} \right) \cdots \left( 1-\frac{1}{p_k} \right),
    \end{align*}
    as required.
  \item The prime divisors of $mn$ are either prime divisors of $m$ or $n$. If
    they are prime divisors of both, then they divide $d=(m,\,n)$. Then
    \[
      \frac{\phi(mn)}{mn} = \prod_{p \mid mn}\left( 1-\frac{1}{p} \right) =
      \frac{\prod_{p \mid m}\left( 1-\frac{1}{p} \right)\prod_{p \mid n}\left(
          1-\frac{1}{p} \right)}{\prod_{p \mid d}\left( 1-\frac{1}{p} \right)}
      = \frac{\frac{\phi(m)}{m}\frac{\phi(n)}{n}}{\frac{\phi(d)}{d}}
    \]
    which suffices to show the result.
  \item We are given $b=ac$. If $a=b$ or $b=c$, the result is trivial. Hence,
    assume $1 < a,\,c < b$. From part (c), we have
    \[
      \phi(b) = \phi(ac) = \phi(a)\phi(c)\frac{(a,\,c)}{\phi((a,\,c))} =
      (a,\,c)\phi(a)\frac{\phi(c)}{\phi((a,\,c))}.
    \]
    Now we use induction. The result is trivial for $b=1$. Suppose $b>1$ and the
    result holds for all smaller positive integers. Then it holds for $c$, i.e.,
    $(a,\,c) \mid c$ so that $\phi((a,\,c)) \mid \phi(c)$. Hence, the
    right-hand-side is a multiple of $\phi(a)$, so that $\phi(a) \mid \phi(b)$.
  \item For $n=p_1^{e_1} \cdots p_k^{e_k}$, we have
    \[
      \phi(n) = n\prod_{p \mid n}\left( 1-\frac{1}{p} \right) =
      \frac{n}{\prod_{p \mid n}p}\prod_{p \mid n}(p-1).
    \]
    If there are $r$ odd prime divisors of $n$, then there are $r$ even factors
    $p-1$ in $\phi(n)$. Thus, $2^r \mid \phi(n)$.
  \end{enumerate}
}

An integer $a$ is a quadratic residue mod the odd prime $p$ if there is a
solution $x$ to $x^2 \equiv q \pmod{p}$. The integer $a$ is a quadratic
nonresidue mod $p$ if there is no such $x$. The Legendre symbol $(\frac{a}{p})$
is defined as
\[
  \left( \frac{a}{p} \right) =
  \begin{cases}
    +1 & \text{if $a$ is a quadratic residue mod $p$,} \\
    -1 & \text{if $a$ is a quadratic nonresidue mod $p$.}
  \end{cases}
\] \\

\noindent{\bf Question 11.}
\begin{enumerate*}[{\bf (a)}]
\item Show that if $a$ is a quadratic residue mod the odd prime $p$, then there
  are exactly two incongruent solutions.
\item Show that $(p-1)!$ is congruent to $-\left( \frac{a}{p}
  \right)a^{(p+1)/2}$ modulo $p$.
\end{enumerate*}

\blue{
  \begin{enumerate}[{\bf (a)}]
  \item We are given that $x^2 \equiv \pmod{p}$ for some $x$. Suppose there is
    another such solution $y$, i.e., $y^2 \equiv a \pmod{p}$. Then
    $x^2-y^2=(x+y)(x-y) \equiv 0 \pmod{p}$. Therefore, either either $p \mid
    x+y$ or $p \mid x-y$. In the first case $y \equiv -x \pmod{p}$; in the
    second, $y \equiv x \pmod{p}$. This shows there are at most two incongruent
    solutions. To see there are at least 2 solutions, note that $(-x)^2 \equiv a
    \pmod{p}$ whenever $x^2 \equiv a \pmod{p}$.
  \item If $a$ is a quadratic residue and suppose $0 \leqq x_1 < p$ is such that
    $x_1^2 = a \pmod{p}$. We group $x_1$ and $p-x_1 \equiv -x_1$ together. For
    every other residue representative $x$, there is a associate $x'$ to $x$ for
    which $xx' = \equiv a \pmod{p}$. There are $(p-3)/2$ such pairings. Now
    $x_1(-x_1) \equiv -a$ and $xx' \equiv a$. Thus $(p-1)! \equiv
    (-a)a^{(p-3)/2} \equiv -a^{(p-1)/2} \pmod{p}$. \\
    If $a$ is a quadratic nonresidue, then each nonzero residue $x$ can be
    uniquely paired with a residue $x'$ such that $xx' \equiv a \pmod{p}$. There
    are $(p-1)/2$ such pairings, hence $(p-1)! \equiv a^{(p-1)/2} \pmod{p}$. \\
    Putting things together, we have that
    \[
      (p-1)! \equiv -\left( \frac{a}{p} \right)a^{(p-1)/2} \pmod{p}.
    \]
  \end{enumerate}
}

\end{document}

