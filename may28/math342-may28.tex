\documentclass[a4paper,11pt]{article}

\usepackage[utf8]{inputenc}
\usepackage[english]{babel}
\usepackage{amssymb, amsmath, amsthm, mathrsfs}
\usepackage[left=1.0in,right=1.0in,top=1.0in,bottom=1.0in]{geometry}
\usepackage[T1]{fontenc}
\usepackage{array}
\usepackage{longtable}
\usepackage{multirow}
\usepackage{calc}
\usepackage[inline,shortlabels]{enumitem}
\usepackage{changepage}
\usepackage{booktabs}
\usepackage{capt-of}
\usepackage{subcaption}
\usepackage[leftcaption]{sidecap}
\usepackage[numbers]{natbib}
\usepackage{times}
\usepackage{titlesec}
\usepackage{xcolor}
\usepackage{lineno}
\usepackage{xpatch}
\xpatchcmd\swappedhead{~}{.~}{}{}
\allowdisplaybreaks

\newtheoremstyle{mythm}
{}                % Space above
{}                % Space below
{\itshape}        % Theorem body font % (default is "\upshape")
{1.5em}                % Indent amount
{\scshape}       % Theorem head font % (default is \mdseries)
{.}               % Punctuation after theorem head % default: no punctuation
{0.5em}               % Space after theorem head
{}                % Theorem head spec
\theoremstyle{mythm}


\newtheorem*{theorem*}{Theorem}
\newtheorem{theorem}{Theorem}
\newtheorem{fact}[theorem]{Fact}
\newtheorem{proposition}[theorem]{Proposition}
\newtheorem{lemma}[theorem]{Lemma}
\newtheorem{corollary}[theorem]{Corollary}
\newtheorem{question}[theorem]{Question}
\newtheorem{result}[theorem]{Result}
\newtheorem{observation}[theorem]{Observation}
\newtheorem{conjecture}[theorem]{Conjecture}

\newtheoremstyle{mydef}
{}                % Space above
{}                % Space below
{}        % Theorem body font % (default is "\upshape")
{1.5em}                % Indent amount
{\scshape}       % Theorem head font % (default is \mdseries)
{.}               % Punctuation after theorem head % default: no punctuation
{0.5em}               % Space after theorem head
{}                % Theorem head spec
\theoremstyle{mydef}

\newtheorem{example}[theorem]{Example}
\newtheorem{definition}[theorem]{Definition}
\newtheorem{remark}[theorem]{Remark}
\newtheorem*{remark*}{Remark}

\makeatletter
\renewenvironment{proof}[1][\proofname]{\par
  \pushQED{\qed}%
  \normalfont \topsep6\p@\@plus6\p@\relax
  \trivlist
\item\relax
  {\hspace{1.5em}\itshape
    #1\@addpunct{.}}\hspace\labelsep\ignorespaces
}{%
  \popQED\endtrivlist\@endpefalse
}
\makeatother

\def\Box{\hskip1ex\vbox{\hrule height0.6pt\hbox{%
      \vrule height1.3ex width0.6pt\hskip0.8ex
      \vrule width0.6pt}\hrule height0.6pt
  }}
\renewcommand{\qed}{\Box}

\newcommand{\red}[1]{\textcolor{red}{#1}}
\newcommand{\blue}[1]{\textcolor{blue}{#1}}
\newcommand{\purple}[1]{\textcolor{magenta}{#1}}
\newcommand{\ddet}{\text{det}}
\renewcommand{\pmod}[1]{\text{ (mod $#1$)}}
\newcommand{\mmod}[2]{#1\text{ mod }#2}
\newcommand{\abs}[1]{\left\vert #1 \right\vert}
\newcommand{\C}{\mathbf{C}}
\newcommand{\Z}{\mathbf{Z}}
\newcommand{\N}{\mathbf{N}}
\newcommand{\LL}{\mathscr{G}}
\newcommand{\z}{\mathbin{\ooalign{$\hidewidth i \hidewidth$\cr$\phantom{+}$}}}
\newcommand{\y}{\mathbin{\ooalign{$\hidewidth j \hidewidth$\cr$\phantom{+}$}}}
\newcommand{\gf}{\text{GF}}

\newcolumntype{R}{>{\scriptsize}r}
\newcolumntype{L}{>{\scriptsize}l}
\newcolumntype{C}{>{\scriptsize}c}

\renewcommand{\citenumfont}[1]{\textbf{#1}}
\renewcommand{\bibnumfmt}[1]{\textbf{#1.}}

\titleformat{\section}{\normalfont\Large\bfseries\centering}{\thesection.}{0.5em}{}
\titleformat{\subsection}{\normalfont\bfseries}{\thesubsection.}{0.5em}{}

\newenvironment{myabstract}{\vspace{1em}\begin{adjustwidth}{3em}{3em}\begin{small}\textbf{Abstract.}}{\end{small}\end{adjustwidth}\vspace{1em}}
\newenvironment{mykeywords}{\vspace{1em}\begin{adjustwidth}{3em}{3em}\begin{small}\textbf{Keywords.}}{\end{small}\end{adjustwidth}\vspace{1em}}

\DeclareCaptionLabelSeparator{custom}{.}
\DeclareCaptionLabelFormat{custom}
{%
  \textsc{#1 #2}
}
\DeclareCaptionFormat{custom}
{%
  #1#2 #3
}
\captionsetup
{
  format=custom,%
  labelformat=custom,%
  labelsep=custom
}

\begin{document}

\begin{center}
  {\Large\bfseries Math 342 Tutorial} \\
  {\normalsize\bf May 28, 2025}
\end{center}

\noindent{\bf Question 1.} Prove that if $a$ and $b$ are different integers,
then there exist infinitely many positive integers $n$ such that $a+n$ and $b+n$
are coprime. [Hint: Consider linear combinations of $b-a$ and $1-a$ if $a<b$.] \\

\noindent{\bf Question 2.} Prove that every integer $>6$ can be represented as a
sum of two integers $>1$ which are coprime. [Hint. Consider the three cases
$n=4k \pm 1$, $n=4k$, and $n=4k+2$ seperately, and write the summands interms of
$k$]. \\

\noindent{\bf Question 3.} An integer $n$ is {\it powerful} if, whenever a prime
$p$ divides $n$, $p^2$ divides $n$. Show that every powerful integer $n$ can be
written as the product of a perfect square and a perfect cube. \\

\noindent{\bf Question 4.} Show that $(a,\,b) \mid [a,\,b]$. When does
$(a,\,b)=[a,\,b]$?  \\

\noindent{\bf Question 5.} Show that if $a,b,c > 0$, then
\[
  (a,\,b,\,c)[ab,\,ac,\,bc] = abc = (ab,\,ac,\,bc)[a,\,b,\,c].
\]

\noindent{\bf Question 6.} An arithmetic function $f: \N\rightarrow\C$ is {\it
  multiplicative} if $f(mn)=f(m)f(n)$ whenever $(m,\,n)=1$. The summatory
function $F$ of an arithmetic function $f:\N\rightarrow\C$ is defined as
$F(x)=\sum_{d \mid x}f(d)$. The number of divisors functions is defined as
$\tau(x)=\#\{d : d \mid x\}$.
\begin{enumerate*}[{\bf (a)}]
\item Show that every summatory function is multiplicative.
\item Show the number of divisors function is multiplicative.
\item If $n=p_1^{e_1}\cdots p_k^{e_k}$, show that $\tau(n)=(e_1+1) \cdots (e_k+1)$.
\item Prove that for every positive integer $k$, the set of all positive
  integers $n$ whose number of positive integer divisors is divisible by $k$
  contains an infinite arithmetic progression. [Hint: Consider a progression
  defined by a linear combination of consecutive powers of 2, and use part (c).]\\
\end{enumerate*}

\noindent{\bf Question 7.} Prove that there exists infinitely many triplets of
positive integers $x,\,y,\,z$ for which the numbers $x(x+1)$, $y(y+1)$, $z(z+1)$
form an increasing arithmetic progression. [Hint: write $y$ and $z$ as
increasing linear functions of $x$.] \\

\noindent{\bf Question 8.} Prove that for every even $n>6$ there exist primes
$p$ and $q$ such that $(n-p,\,n-q)=1$. \\

\noindent{\bf Question 9.}
\begin{enumerate*}[{\bf (a)}]
\item Prove that for every three consecutive odd integers, one must be divisible
  by 3. [Hint. Write $n=2k+1$ and consider the possible cases for $k \pmod{3}$.]
\item Find all primes which can be represented as both a sum and difference of
  primes. \\
\end{enumerate*}

\noindent{\bf Question 10.} Find all integer solutions $x,\,y$ of the equation
$2x^3+xy-7=0$ and prove that this equation has infinitely many solutions in
positive rationals. [Hint: Use the possible values for $x$ in the first part to
infer a possible form for $x$ in the second part.] \\

\noindent{\bf Question 11.} An astronomer knows that a satellite orbits the
Earth in a period that is an exact multiple of 1 hour that is less than 1 day.
If the astronomer notes that the satellite completes 11 orbits in an interval
that starts when a 24-hour clock reads 0 hours and ends when the clock reads 17
hours, how long is the orbital period of the satellite? \\

\noindent{\bf Question 12.}
\begin{enumerate*}[{\bf (a)}]
\item Let $p$ be an odd prime. Show the congruence $x^2 \equiv 1\pmod{p^k}$ has
  exactly two incongruent solutions, namely, $x \equiv \pm1 \pmod{p^k}$.
\item Show that the congruence $x^2 \equiv 1 \pmod{2^k}$ has exactly four
  ingongruent solutions, namely, $x \equiv \pm1\,\pm(1-2^{k-1}) \pmod{2^k}$,
  when $k>2$. Show there is one when $k=1$ and two when $k=2$.
\end{enumerate*}

\end{document}

