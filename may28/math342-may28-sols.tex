\documentclass[a4paper,11pt]{article}

\usepackage[utf8]{inputenc}
\usepackage[english]{babel}
\usepackage{amssymb, amsmath, amsthm, mathrsfs}
\usepackage[left=1.0in,right=1.0in,top=1.0in,bottom=1.0in]{geometry}
\usepackage[T1]{fontenc}
\usepackage{array}
\usepackage{longtable}
\usepackage{multirow}
\usepackage{calc}
\usepackage[inline,shortlabels]{enumitem}
\usepackage{changepage}
\usepackage{booktabs}
\usepackage{capt-of}
\usepackage{subcaption}
\usepackage[leftcaption]{sidecap}
\usepackage[numbers]{natbib}
\usepackage{times}
\usepackage{titlesec}
\usepackage{xcolor}
\usepackage{lineno}
\usepackage{xpatch}
\xpatchcmd\swappedhead{~}{.~}{}{}
\allowdisplaybreaks

\newtheoremstyle{mythm}
{}                % Space above
{}                % Space below
{\itshape}        % Theorem body font % (default is "\upshape")
{1.5em}                % Indent amount
{\scshape}       % Theorem head font % (default is \mdseries)
{.}               % Punctuation after theorem head % default: no punctuation
{0.5em}               % Space after theorem head
{}                % Theorem head spec
\theoremstyle{mythm}


\newtheorem*{theorem*}{Theorem}
\newtheorem{theorem}{Theorem}
\newtheorem{fact}[theorem]{Fact}
\newtheorem{proposition}[theorem]{Proposition}
\newtheorem{lemma}[theorem]{Lemma}
\newtheorem{corollary}[theorem]{Corollary}
\newtheorem{question}[theorem]{Question}
\newtheorem{result}[theorem]{Result}
\newtheorem{observation}[theorem]{Observation}
\newtheorem{conjecture}[theorem]{Conjecture}

\newtheoremstyle{mydef}
{}                % Space above
{}                % Space below
{}        % Theorem body font % (default is "\upshape")
{1.5em}                % Indent amount
{\scshape}       % Theorem head font % (default is \mdseries)
{.}               % Punctuation after theorem head % default: no punctuation
{0.5em}               % Space after theorem head
{}                % Theorem head spec
\theoremstyle{mydef}

\newtheorem{example}[theorem]{Example}
\newtheorem{definition}[theorem]{Definition}
\newtheorem{remark}[theorem]{Remark}
\newtheorem*{remark*}{Remark}

\makeatletter
\renewenvironment{proof}[1][\proofname]{\par
  \pushQED{\qed}%
  \normalfont \topsep6\p@\@plus6\p@\relax
  \trivlist
\item\relax
  {\hspace{1.5em}\itshape
    #1\@addpunct{.}}\hspace\labelsep\ignorespaces
}{%
  \popQED\endtrivlist\@endpefalse
}
\makeatother

\def\Box{\hskip1ex\vbox{\hrule height0.6pt\hbox{%
      \vrule height1.3ex width0.6pt\hskip0.8ex
      \vrule width0.6pt}\hrule height0.6pt
  }}
\renewcommand{\qed}{\Box}

\newcommand{\red}[1]{\textcolor{red}{#1}}
\newcommand{\blue}[1]{\textcolor{blue}{#1}}
\newcommand{\purple}[1]{\textcolor{magenta}{#1}}
\newcommand{\ddet}{\text{det}}
\renewcommand{\pmod}[1]{\text{ (mod $#1$)}}
\newcommand{\mmod}[2]{#1\text{ mod }#2}
\newcommand{\abs}[1]{\left\vert #1 \right\vert}
\newcommand{\C}{\mathbf{C}}
\newcommand{\Z}{\mathbf{Z}}
\newcommand{\N}{\mathbf{N}}
\newcommand{\LL}{\mathscr{G}}
\newcommand{\z}{\mathbin{\ooalign{$\hidewidth i \hidewidth$\cr$\phantom{+}$}}}
\newcommand{\y}{\mathbin{\ooalign{$\hidewidth j \hidewidth$\cr$\phantom{+}$}}}
\newcommand{\gf}{\text{GF}}

\newcolumntype{R}{>{\scriptsize}r}
\newcolumntype{L}{>{\scriptsize}l}
\newcolumntype{C}{>{\scriptsize}c}

\renewcommand{\citenumfont}[1]{\textbf{#1}}
\renewcommand{\bibnumfmt}[1]{\textbf{#1.}}

\titleformat{\section}{\normalfont\Large\bfseries\centering}{\thesection.}{0.5em}{}
\titleformat{\subsection}{\normalfont\bfseries}{\thesubsection.}{0.5em}{}

\newenvironment{myabstract}{\vspace{1em}\begin{adjustwidth}{3em}{3em}\begin{small}\textbf{Abstract.}}{\end{small}\end{adjustwidth}\vspace{1em}}
\newenvironment{mykeywords}{\vspace{1em}\begin{adjustwidth}{3em}{3em}\begin{small}\textbf{Keywords.}}{\end{small}\end{adjustwidth}\vspace{1em}}

\DeclareCaptionLabelSeparator{custom}{.}
\DeclareCaptionLabelFormat{custom}
{%
  \textsc{#1 #2}
}
\DeclareCaptionFormat{custom}
{%
  #1#2 #3
}
\captionsetup
{
  format=custom,%
  labelformat=custom,%
  labelsep=custom
}

\begin{document}

\begin{center}
  {\Large\bfseries Math 342 Tutorial} \\
  {\normalsize\bf May 28, 2025}
\end{center}

\noindent{\bf Question 1.} Prove that if $a$ and $b$ are different integers,
then there exist infinitely many positive integers $n$ such that $a+n$ and $b+n$
are coprime. [Hint: Consider linear combinations of $b-a$ and $1-a$ if $a<b$.] \\

\blue{Assume that $a<b$, and let $n=(b-a)k+(1-a)$. We have that $n>0$ for large
  enough $k$. Now $a+n=(b-a)k+1$ and $b+n=(b-a)(k+1)+1$, hence $a+n,\,b+n > 0$.
  If we had $d \mid a+n,\,b+n$, we would have $d \mid a-b$ and so $d \mid 1$
  since $d \mid a+n$. Thus, $d=1$ and $(a+n,\,b+n)=1$.} \\

\red{Here is an additional solution pointed out by several in the class. Let $p$
be a prime greater than $b$, and write $n=p-b > 0$. Since $a<b$, we have $a+n <
b+n=p$. As $p$ is prime, $(a+n,\,b+n)=1$. Since there are infinitely many primes
greater than $b$, there are infinitely many $n$ satisfying the requirement. This
solution is valid here because we only asked for infinitely many such $n$. If
instead we were required to provide a closed form solution for each $n$, then we
would need to provide something like the first solution.} \\

\noindent{\bf Question 2.} Prove that every integer $>6$ can be represented as a
sum of two integers $>1$ which are coprime. [Hint. Consider the three cases
$n=4k \pm 1$, $n=4k$, and $n=4k+2$ seperately, and write the summands interms of
$k$]. \\

\blue{The easy case is if $n>6$ is odd, for then $n=2+(n-2)$ and clearly
  $(2,\,n-2)=1$. Next, consider the case $n=4k$. Then $n=(2k+1)+(2k-1)$ where
  certainly $(2k+1,\,2k-1)=1$ and $2k+1 > 2k-1 > 1$ as $k>1$. Finally, consider
  the case $n=4k+2$. We have that $4k+2=(2k+3)+(2k-1)$. If $d \mid 2k+3,\,2k-1$,
  then $d \mid (2k+3)-(2k-1)=4$. Since $d$ must be odd, we have that $d=1$.
  Observe further that $2k+3>2k-1>1$ as $k>1$.} \\

\noindent{\bf Question 3.} An integer $n$ is {\it powerful} if, whenever a prime
$p$ divides $n$, $p^2$ divides $n$. Show that every powerful integer $n$ can be
written as the product of a perfect square and a perfect cube. \\

\blue{The exponents in the nonredundant prime power factorization of $a$ are all
at least $2$. If $a$ is a square, we're done as we can write $a=u^21^3$.
Therefore, assume $a$ is not a perfect square. Let $p_1,\dots,\,p_k$ be the
primes appearing with even exponent, and let $q_1,\dots,\,q_l$ be the primes
with odd exponent. Since the odd exponents are at least 3, we can write
\[
  a=p_1^{2e_1} \cdots p_k^{e_k}q_1^{2f_1+3} \cdots q_l^{2f_l+3} \quad\mbox{for
    $e_i,\,f_i \geqq 0$.}
\]
But then
\[
  a=(p_1^{e_1} \cdots p_k^{e_k}q_1^{f_1} \cdots q_l^{f_l})^2(q_1 \cdots q_l)^3
\]
as required.} \\

\noindent{\bf Question 4.} Show that $(a,\,b) \mid [a,\,b]$. When does
$(a,\,b)=[a,\,b]$?  \\

\blue{We have $(a,\,b) \mid a$ and $a \mid [a,\,b]$, hence $(a,\,b) \mid
  [a,\,b]$. Let $a=p_1^{r_1} \cdots p_k^{r_k}$ and $b=p_1^{s_1} \cdots
  p_k^{s_k}$. Then $(a,\,b)=[a,\,b]$ if and only if
  $\min\{r_i,\,s_i\}=\max\{r_i,\,s_i\}$. If $r_i < s_i$, then $\max\{r_i,\,s_i\}
  = s_i \neq r_i = \min\{r_i,\,s_i\}$ which contradicts our assumption that
  $\max\{r_I,\,s_i\}=\min\{r_i,\,s_i\}$. Thus, $r_i \geqq s_i$. Similarly,
  however, $s_i < r_i$ cannot happen. We are left with $r_i=s_i$ which shows
  that $a=b$.} \\

\noindent{\bf Question 5.} Show that if $a,b,c > 0$, then
\[
  (a,\,b,\,c)[ab,\,ac,\,bc] = abc = (ab,\,ac,\,bc)[a,\,b,\,c].
\]

\blue{Let $a\,b,\,c$ have prime factorizations $a=p_1^{r_1} \cdots p_k^{r_k}$,
  $b=p_1^{s_1} \cdots p_k^{s_k}$, $c=p_1^{t_1} \cdots p_k^{t_k}$. Then
  $p_i^{r_i+s_i+t_i} \Vert abc$, but $p_i^{\min\{r_i,s_i,t_i\}} \Vert
  (a,\,b,\,c)$ and $p_i^{r_i+s_i+t_i-\min\{r_i,s_i,t_i\}} \Vert [ab,\,ac,\,bc]$,
and
$p_i^{r_i+s_i+t_i-\min\{r_i,s_i,t_i\}}p_i^{\min\{r_i,s_i,t_i\}}=p^{r_i+s_i+t_i}$.
Therefore, $(a,\,b,\,c)[ab,\,ac,\,bc]=abc$. One may similarly show that
$[a,\,b,\,c](ab,\,ac,\,bc)=abc$.} \\

\noindent{\bf Question 6.} An arithmetic function $f: \N\rightarrow\C$ is {\it
  multiplicative} if $f(mn)=f(m)f(n)$ whenever $(m,\,n)=1$. The summatory
function $F$ of an arithmetic function $f:\N\rightarrow\C$ is defined as
$F(x)=\sum_{d \mid x}f(d)$. The number of divisors functions is defined as
$\tau(x)=\#\{d : d \mid x\}$.
\begin{enumerate*}[{\bf (a)}]
\item Show that every summatory function of a multiplicative function is
  multiplicative.
\item Show the number of divisors function is multiplicative.
\item If $n=p_1^{e_1}\cdots p_k^{e_k}$, show that $\tau(n)=(e_1+1) \cdots (e_k+1)$.
\item Prove that for every positive integer $k$, the set of all positive
  integers $n$ whose number of positive integer divisors is divisible by $k$
  contains an infinite arithmetic progression. [Hint: Consider a progression
  defined by a linear combination of consecutive powers of 2, and use part (c).]
\end{enumerate*}

\blue{
  \begin{enumerate}[(a)]
  \item Let $m,n > 0$ be such that $(m,\,n)=1$. Thus, every divisor $d$ of
    $mn$ can be written as $d=d'd''$ where $d' \mid m$ and $d'' \mid n$ with
    $(d',\,d'')=1$. Observe, therefore, that
    \[
      F(mn) = \sum_{d \mid mn}f(d) = \sum_{\begin{smallmatrix}d' \mid m \\ d''
        \mid n\end{smallmatrix}} f(d'd'') = \sum_{\begin{smallmatrix}d' \mid m
          \\ d'' \mid n\end{smallmatrix}}f(d')f(d'') = \sum_{d' \mid
        m}f(d')\sum_{d'' \mid n}f(d'') = F(m)F(n).
    \]
    We have shown that $F$ is multiplicative.
  \item Let $id(x)=1$ be the constant function which maps every integer $x$ to
    1. Certainly, $id$ is multiplicative. Observe $\tau(a)=\sum_{d \mid
      x}id(x)=\sum_{d \mid x}1=\#\{d : d \mid x\}$. From part (a), we have that
    $\tau$ is therefore multiplicative.
  \item Consider the prime power $p^e$, $e \geqq 0$. The divisors of $p^e$ are
    $1,\,p,\dots,\,p^e$. So, $p^e$ has $e+1$ divisors. Writing $n=p_1^{e_1}
    \cdots p_k^{e_k}$, and from part (b), we see that
    \[
      \tau(n)=\tau(p_1^{e_1}) \cdots \tau(p_k^{e_k}) = (e_1+1)\cdots(e_k+1).
    \]
  \item We consider the infinite progression $2^kn+2^{k-1}$ for $n \geqq 0$. We
    can write the general term as $2^{k-1}(2n+1)$ where obviously $2^{k-1} \Vert
    2^{k-1}(2n+1)$. From parts (b) and (c) above,
    $\tau(2^kn+2^{k-1})=\tau(2^{k-1})\tau(2n+1)=k\tau(2n+1)$. We have,
    therefore, an infinite progression satisfying the required property. \\
  \end{enumerate}
}

\noindent{\bf Question 7.} Prove that there exists infinitely many triplets of
positive integers $x,\,y,\,z$ for which the numbers $x(x+1)$, $y(y+1)$, $z(z+1)$
form an increasing arithmetic progression. [Hint: write $y$ and $z$ as
increasing linear functions of $x$.] \\

\blue{Let $x>0$ be arbitrary, and define $y=5x+2$ and $z=7x+3$. Then
  \[
    y(y+1)-x(x+1) = z(z+1)-y(y+1) = 24x^224x++6 > 0\quad\mbox{since $x>0$.}
  \]
}

\noindent{\bf Question 8.} Prove that for every even $n>6$ there exist primes
$p$ and $q$ such that $(n-p,\,n-q)=1$. \\

\blue{It suffices to take $p=3$ and $q=5$. If $n>6$ is even, the we have $n-1
  \geqq 6$ and $p < q < n-1$. The numbers $n-p=n-3$ and $n-q=n-5$ as consecutive
odd numbers, are relatively prime.} \\

\noindent{\bf Question 9.}
\begin{enumerate*}[{\bf (a)}]
\item Prove that for every three consecutive odd integers, one must be divisible
  by 3. [Hint. Write $n=2k+1$ and consider the possible cases for $k \pmod{3}$.]
\item Find all primes which can be represented as both a sum and difference of
  primes.
\end{enumerate*}

\blue{
  \begin{enumerate}[(a)]
  \item Let $k$ be an arbitrary integer, and consider the consecutive integers
    $2k+1$, $2k+3$, and $2k+5$. If $k \equiv 0 \pmod{3}$, then $2k+3 \equiv 0
    \pmod{3}$. If $k \equiv 1\pmod{3}$, then $2k+1 \equiv 0\pmod{3}$. If $k
    \equiv 2\pmod{3}$, then $2k+5 \equiv 0\pmod{3}$. In every case, we have that
    one of $2k+1$, $2k+3$, and $2k+5$ is divisible by 3. Since $k$ was
    arbitrary, the result follows.
  \item Suppose that $r$ is an arbitrary prime that can be represented
    simultaneously as a sum and difference of two pairs of prime numbers.
    Certainly, $r \neq 2$, hence $r>2$ must be odd. Therefore, one of the primes
    from each pair of representing primes must be even, i.e., we have that
    $r=p+2=q-2$ for some odd primes $p,\,q$. But then $p,\,r,\,q$ are three
    consecutive odd prime. From part (a), $(p,\,r,\,q)=(3,\,5,\,7)$ are the only
    three consecutive odd primes so that $r=5$ is the only solution. \\
  \end{enumerate}
}

\noindent{\bf Question 10.} Find all integer solutions $x,\,y$ of the equation
$2x^3+xy-7=0$ and prove that this equation has infinitely many solutions in
positive rationals. [Hint: Use the possible values for $x$ in the first part to
infer a possible form for $x$ in the second part.] \\

\blue{Since $x(2x^2+y)=7$, we have that $x=\pm1,\,\pm7$. Upon substituting
  these values for $x$, we find that $y=5,\,-97,\,-9,\,-99$ as the possible
  values for $y$.}

\blue{Let $n>5$ be arbitrary, and let $x=7/n$ so that $y=\frac{n-98}{n^2}$.
  These are rational and positive solutions to $2x^3+xy-7=0$.} \\

\noindent{\bf Question 11.} An astronomer knows that a satellite orbits the
Earth in a period that is an exact multiple of 1 hour that is less than 1 day.
If the astronomer notes that the satellite completes 11 orbits in an interval
that starts when a 24-hour clock reads 0 hours and ends when the clock reads 17
hours, how long is the orbital period of the satellite? \\

\blue{This is equivalent to finding solutions to $11x \equiv 17 \pmod{24}$. By
  Theorem~4.11, there is a unique solution given by $x \equiv 19 \pmod{24}$. So
  the satellite orbits the Earth every 19 hours.} \\

\noindent{\bf Question 12.}
\begin{enumerate*}[{\bf (a)}]
\item Let $p$ be an odd prime. Show the congruence $x^2 \equiv 1\pmod{p^k}$ has
  exactly two incongruent solutions, namely, $x \equiv \pm1 \pmod{p^k}$.
\item Show that the congruence $x^2 \equiv 1 \pmod{2^k}$ has exactly four
  incongruent solutions, namely, $x \equiv \pm1\,\pm(1-2^{k-1}) \pmod{2^k}$,
  when $k>2$. Show there is one when $k=1$ and two when $k=2$.
\end{enumerate*}

\blue{
  \begin{enumerate}[(a)]
  \item If $x^2 \equiv 1 \pmod{p^k}$, then $x^2-1 = (x-1)(x+1) \equiv 0
    \pmod{p^k}$. Therefore, $p^k \mid (x+1)(x-1)$. Since $(x+1)-(x-1)=2$ and $p$
    is odd, we have that $p$ can divide at most one of $x+1$ and $x-1$.
    Therefore, either $p^k \mid x+1$ or $p^k \mid x-1$. In particular, $p \equiv
    \pm1\pmod{p^k}$.
  \item As in (a), we have $2^k \mid (x+1)(x-1)$. Since $(x+1)-(x-1)=2$, we have
    either $2^{k-1} \mid x+1$ and $2 \mid x-1$, or we have $2^{k-1} \mid x-1$
    and $2 \mid x+1$. Hence, $x=t2^{k-1} \pm 1$, where $t \in \Z$. Modulo $2^k$,
    there are four solutions given by $t=0\text{ or }1$, i.e., $x \equiv
    \pm1\text{ or }\pm(1+2^{k-1}) \pmod{2^k}$. \\
    When $k=1$, the only solution is $x \equiv 1 \pmod{2}$. When $k=2$, the only
    solutions are $x \equiv \pm1\pmod{4}$.
  \end{enumerate}
}

\end{document}

