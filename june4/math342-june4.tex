\documentclass[a4paper,11pt]{article}

\usepackage[utf8]{inputenc}
\usepackage[english]{babel}
\usepackage{amssymb, amsmath, amsthm, mathrsfs}
\usepackage[left=1.0in,right=1.0in,top=1.0in,bottom=1.0in]{geometry}
\usepackage[T1]{fontenc}
\usepackage{array}
\usepackage{longtable}
\usepackage{multirow}
\usepackage{calc}
\usepackage[inline,shortlabels]{enumitem}
\usepackage{changepage}
\usepackage{booktabs}
\usepackage{capt-of}
\usepackage{subcaption}
\usepackage[leftcaption]{sidecap}
\usepackage[numbers]{natbib}
\usepackage{times}
\usepackage{titlesec}
\usepackage{xcolor}
\usepackage{lineno}
\usepackage{xpatch}
\xpatchcmd\swappedhead{~}{.~}{}{}
\allowdisplaybreaks

\newtheoremstyle{mythm}
{}                % Space above
{}                % Space below
{\itshape}        % Theorem body font % (default is "\upshape")
{1.5em}                % Indent amount
{\scshape}       % Theorem head font % (default is \mdseries)
{.}               % Punctuation after theorem head % default: no punctuation
{0.5em}               % Space after theorem head
{}                % Theorem head spec
\theoremstyle{mythm}


\newtheorem*{theorem*}{Theorem}
\newtheorem{theorem}{Theorem}
\newtheorem{fact}[theorem]{Fact}
\newtheorem{proposition}[theorem]{Proposition}
\newtheorem{lemma}[theorem]{Lemma}
\newtheorem{corollary}[theorem]{Corollary}
\newtheorem{question}[theorem]{Question}
\newtheorem{result}[theorem]{Result}
\newtheorem{observation}[theorem]{Observation}
\newtheorem{conjecture}[theorem]{Conjecture}

\newtheoremstyle{mydef}
{}                % Space above
{}                % Space below
{}        % Theorem body font % (default is "\upshape")
{1.5em}                % Indent amount
{\scshape}       % Theorem head font % (default is \mdseries)
{.}               % Punctuation after theorem head % default: no punctuation
{0.5em}               % Space after theorem head
{}                % Theorem head spec
\theoremstyle{mydef}

\newtheorem{example}[theorem]{Example}
\newtheorem{definition}[theorem]{Definition}
\newtheorem{remark}[theorem]{Remark}
\newtheorem*{remark*}{Remark}

\makeatletter
\renewenvironment{proof}[1][\proofname]{\par
  \pushQED{\qed}%
  \normalfont \topsep6\p@\@plus6\p@\relax
  \trivlist
\item\relax
  {\hspace{1.5em}\itshape
    #1\@addpunct{.}}\hspace\labelsep\ignorespaces
}{%
  \popQED\endtrivlist\@endpefalse
}
\makeatother

\def\Box{\hskip1ex\vbox{\hrule height0.6pt\hbox{%
      \vrule height1.3ex width0.6pt\hskip0.8ex
      \vrule width0.6pt}\hrule height0.6pt
  }}
\renewcommand{\qed}{\Box}

\newcommand{\red}[1]{\textcolor{red}{#1}}
\newcommand{\blue}[1]{\textcolor{blue}{#1}}
\newcommand{\purple}[1]{\textcolor{magenta}{#1}}
\newcommand{\ddet}{\text{det}}
\renewcommand{\pmod}[1]{\text{ (mod $#1$)}}
\newcommand{\mmod}[2]{#1\text{ mod }#2}
\newcommand{\abs}[1]{\left\vert #1 \right\vert}
\newcommand{\C}{\mathbf{C}}
\newcommand{\Z}{\mathbf{Z}}
\newcommand{\N}{\mathbf{N}}
\newcommand{\LL}{\mathscr{G}}
\newcommand{\z}{\mathbin{\ooalign{$\hidewidth i \hidewidth$\cr$\phantom{+}$}}}
\newcommand{\y}{\mathbin{\ooalign{$\hidewidth j \hidewidth$\cr$\phantom{+}$}}}
\newcommand{\gf}{\text{GF}}

\newcolumntype{R}{>{\scriptsize}r}
\newcolumntype{L}{>{\scriptsize}l}
\newcolumntype{C}{>{\scriptsize}c}

\renewcommand{\citenumfont}[1]{\textbf{#1}}
\renewcommand{\bibnumfmt}[1]{\textbf{#1.}}

\titleformat{\section}{\normalfont\Large\bfseries\centering}{\thesection.}{0.5em}{}
\titleformat{\subsection}{\normalfont\bfseries}{\thesubsection.}{0.5em}{}

\newenvironment{myabstract}{\vspace{1em}\begin{adjustwidth}{3em}{3em}\begin{small}\textbf{Abstract.}}{\end{small}\end{adjustwidth}\vspace{1em}}
\newenvironment{mykeywords}{\vspace{1em}\begin{adjustwidth}{3em}{3em}\begin{small}\textbf{Keywords.}}{\end{small}\end{adjustwidth}\vspace{1em}}

\DeclareCaptionLabelSeparator{custom}{.}
\DeclareCaptionLabelFormat{custom}
{%
  \textsc{#1 #2}
}
\DeclareCaptionFormat{custom}
{%
  #1#2 #3
}
\captionsetup
{
  format=custom,%
  labelformat=custom,%
  labelsep=custom
}

\begin{document}

\begin{center}
  {\Large\bfseries Math 342 Tutorial} \\
  {\normalsize\bf June 4, 2025}
\end{center}

\noindent{\bf Question 1.}
\begin{enumerate*}[{\bf (a)}]
\item What is the remainder of when $5^{100}$ is divided by $7$?
\item What is the remainder when $18!$ is divided by $437$? \\
\end{enumerate*}

\noindent{\bf Question 2.} Show that if $p$ is an odd prime, then $2(p-3) \equiv
-1 \pmod{p}$. \\

\noindent{\bf Question 3.} Show that if $n$ is a composite integer with $n \neq
4$, then $(n-1)! \equiv 0 \pmod{n}$. \\

\noindent{\bf Question 4.} Show that $1^{p-1}+2^{p-1}+\cdots+(p-1)^{p-1} \equiv
-1 \pmod{p}$. \\

\noindent{\bf Question 5.}
\begin{enumerate*}[{\bf (a)}]
\item Let $p$ be a prime. Show that $\binom{p}{n} \equiv 0 \pmod{p}$ whenever
  $1 < n < p$.
\item Show that $(x_1+\cdots+x_n)^{p^e} \equiv x_1^{p^e}+\cdots+x_n^{p^e} \pmod{p}$.
\item Show that $1^p+2^p+\cdots+(p-1)^p \equiv 0 \pmod{p}$. \\
\end{enumerate*}

\noindent{\bf Question 6.}
\begin{enumerate*}[{\bf (a)}]
\item Show that if $p$ and $q$ are distinct primes, then $p^{q-1}+q^{p-1} \equiv
  1 \pmod{pq}$.
\item Show that if $p$ is prime, then $p \mid (a^p+(p-1)!a)$.
\item Show that if $p$ is an odd prime, then $1^22^2 \cdots (p-4)^2(p-2)^2$ is
  equivalent to $(-1)^{(p+1)/2}$ modulo $p$. \\
\end{enumerate*}

The questions 7 and 9 each show by elementary means that $\phi$ is a
multiplicative arithmetical function (your textbook contains a third). Next
week, we will derive an algebraic proof using the Chinese Remainder Theorem.
Many of the results you are learning via elementary means admit more natural
interpretations in more general algebraic settings. We will see that the main
properties of the Euler Totient Function are easier understood in this more
general setting. \\

\noindent{\bf Question 7.}
\begin{enumerate*}[{\bf (a)}]
\item Suppose that $(m,\,m')=1$, and that $a$ runs through a complete system of
  residues modulo $m$, and that $a'$ runs through a complete system of integers
  modulo $m'$. Show that $a'm+am'$ runs through a complete set of residues
  modulo $mm'$.
\item Show that $\phi$ is multiplicative, that is, if $(m,\,m')=1$, then
  $\phi(mm')=\phi(m)\phi(m')$. [Hint: Consider $(a'm+am',\,mm')$ modulo both $m$
  and $m'$.] \\
\end{enumerate*}

Given two arithmetical functions $f$ and $g$, we define their Dirichlet product
$f*g$ as $(f*g)(n)=\sum_{d \mid n}f(n)g(\frac{n}{d})$. The function $\iota$ is
defined by $\iota(n)=\lfloor \frac{1}{n} \rfloor$. We call $\iota$ the
arithmetic identity. \\

\noindent{\bf Question 8.} Let $f$ and $g$ be two arithmetical functions. Show
the following.
\begin{enumerate*}[{\bf (a)}]
\item $f*g = g*f$.
\item $(f*g)*h=f*(g*h)$.
\item $f*\iota = \iota*f=f$.
\item $g$ is the arithmetic inverse of $f$ if $f*g=\iota$. Show the arithmetic
  function $f$ has an inverse if and only if $f(1) \neq 0$. Show that if $f$ has
  an arithmetic inverse, then it is unique.
\item If $h$ is the arithmetic inverse of $g$, then $(f*g)*h=f$.
\item If $f$ and $g$ are multiplicative, then so is $f*g$. \\
\end{enumerate*}

The M\"{o}bius function $\mu(n)$ is defined as follows: $\mu(1)=1$; if
$n=p_1^{e_1} \cdots p^{e_k}$, then $\mu(n)=(-1)^k$ if $e_1=\cdots=e_k=1$;
$\mu(n)=0$ otherwise. \\

\noindent{\bf Question 9.} Show the following.
\begin{enumerate*}[{\bf (a)}]
\item $\mu$ is multiplicative.
\item $\mu(n)=\sum_{d \mid n}\lfloor \frac{1}{n} \rfloor$.
\item Show that $\phi$ is multiplicative using part (b). [Hint: Argue that you
  can write $\phi(n)=\sum_{k = 1}^n\lfloor \frac{1}{(n,\,k)} \rfloor$] \\
\end{enumerate*}

\noindent{\bf Question 10.} Show the following properties of $\phi$.
\begin{enumerate}[{\bf (a)}]
\item $\phi(p^e)=p^e-p$ for every prime $p$ and $e>0$.
\item If $n=p_1^{e_1} \cdots p_k^{e_k}$, then $\phi(n)=n\prod_{p \mid
    n}(1-\frac{1}{p})$.
\item $\phi(mn)=\phi(m)\phi(n)d/\phi(d)$ where $d=(n,\,m)$.
\item $a \mid b$ implies $\phi(a) \mid \phi(b)$.
\item $\phi(n)$ is even for $n \geqq 3$. Moreover, if $n$ has $r$ distinct odd
  prime factors, then $2^r \mid \phi(n)$. \\
\end{enumerate}

An integer $a$ is a quadratic residue mod the odd prime $p$ if there is a
solution $x$ to $x^2 \equiv q \pmod{p}$. The integer $a$ is a quadratic
nonresidue mod $p$ if there is no such $x$. The Legendre symbol $(\frac{a}{p})$
is defined as
\[
  \left( \frac{a}{p} \right) =
  \begin{cases}
    +1 & \text{if $a$ is a quadratic residue mod $p$,} \\
    -1 & \text{if $a$ is a quadratic nonresidue mod $p$.}
  \end{cases}
\] \\

\noindent{\bf Question 11.}
\begin{enumerate*}[{\bf (a)}]
\item Show that if $a$ is a quadratic residue mod the odd prime $p$, then there
  are at most two incongruent solutions.
\item Show that $(p-1)!$ is congruent to $-\left( \frac{a}{p}
  \right)a^{(p+1)/2}$ modulo $p$.
\end{enumerate*}

\end{document}

