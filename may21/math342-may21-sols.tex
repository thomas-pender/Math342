\documentclass[a4paper,11pt]{article}

\usepackage[utf8]{inputenc}
\usepackage[english]{babel}
\usepackage{amssymb, amsmath, amsthm, mathrsfs}
\usepackage[left=1.0in,right=1.0in,top=1.0in,bottom=1.0in]{geometry}
\usepackage[T1]{fontenc}
\usepackage{array}
\usepackage{longtable}
\usepackage{multirow}
\usepackage{calc}
\usepackage[inline,shortlabels]{enumitem}
\usepackage{changepage}
\usepackage{booktabs}
\usepackage{capt-of}
\usepackage{subcaption}
\usepackage[leftcaption]{sidecap}
\usepackage[numbers]{natbib}
\usepackage{times}
\usepackage{titlesec}
\usepackage{xcolor}
\usepackage{lineno}
\usepackage{xpatch}
\xpatchcmd\swappedhead{~}{.~}{}{}
\allowdisplaybreaks

\newtheoremstyle{mythm}
{}                % Space above
{}                % Space below
{\itshape}        % Theorem body font % (default is "\upshape")
{1.5em}                % Indent amount
{\scshape}       % Theorem head font % (default is \mdseries)
{.}               % Punctuation after theorem head % default: no punctuation
{0.5em}               % Space after theorem head
{}                % Theorem head spec
\theoremstyle{mythm}


\newtheorem*{theorem*}{Theorem}
\newtheorem{theorem}{Theorem}
\newtheorem{fact}[theorem]{Fact}
\newtheorem{proposition}[theorem]{Proposition}
\newtheorem{lemma}[theorem]{Lemma}
\newtheorem{corollary}[theorem]{Corollary}
\newtheorem{question}[theorem]{Question}
\newtheorem{result}[theorem]{Result}
\newtheorem{observation}[theorem]{Observation}
\newtheorem{conjecture}[theorem]{Conjecture}

\newtheoremstyle{mydef}
{}                % Space above
{}                % Space below
{}        % Theorem body font % (default is "\upshape")
{1.5em}                % Indent amount
{\scshape}       % Theorem head font % (default is \mdseries)
{.}               % Punctuation after theorem head % default: no punctuation
{0.5em}               % Space after theorem head
{}                % Theorem head spec
\theoremstyle{mydef}

\newtheorem{example}[theorem]{Example}
\newtheorem{definition}[theorem]{Definition}
\newtheorem{remark}[theorem]{Remark}
\newtheorem*{remark*}{Remark}

\makeatletter
\renewenvironment{proof}[1][\proofname]{\par
  \pushQED{\qed}%
  \normalfont \topsep6\p@\@plus6\p@\relax
  \trivlist
\item\relax
  {\hspace{1.5em}\itshape
    #1\@addpunct{.}}\hspace\labelsep\ignorespaces
}{%
  \popQED\endtrivlist\@endpefalse
}
\makeatother

\def\Box{\hskip1ex\vbox{\hrule height0.6pt\hbox{%
      \vrule height1.3ex width0.6pt\hskip0.8ex
      \vrule width0.6pt}\hrule height0.6pt
  }}
\renewcommand{\qed}{\Box}

\newcommand{\red}[1]{\textcolor{red}{#1}}
\newcommand{\blue}[1]{\textcolor{blue}{#1}}
\newcommand{\purple}[1]{\textcolor{magenta}{#1}}
\newcommand{\ddet}{\text{det}}
\renewcommand{\pmod}[1]{\text{ (mod $#1$)}}
\newcommand{\mmod}[2]{#1\text{ mod }#2}
\newcommand{\abs}[1]{\left\vert #1 \right\vert}
\newcommand{\C}{\mathbf{C}}
\newcommand{\Z}{\mathbf{Z}}
\newcommand{\LL}{\mathscr{G}}
\newcommand{\z}{\mathbin{\ooalign{$\hidewidth i \hidewidth$\cr$\phantom{+}$}}}
\newcommand{\y}{\mathbin{\ooalign{$\hidewidth j \hidewidth$\cr$\phantom{+}$}}}
\newcommand{\gf}{\text{GF}}

\newcolumntype{R}{>{\scriptsize}r}
\newcolumntype{L}{>{\scriptsize}l}
\newcolumntype{C}{>{\scriptsize}c}

\renewcommand{\citenumfont}[1]{\textbf{#1}}
\renewcommand{\bibnumfmt}[1]{\textbf{#1.}}

\titleformat{\section}{\normalfont\Large\bfseries\centering}{\thesection.}{0.5em}{}
\titleformat{\subsection}{\normalfont\bfseries}{\thesubsection.}{0.5em}{}

\newenvironment{myabstract}{\vspace{1em}\begin{adjustwidth}{3em}{3em}\begin{small}\textbf{Abstract.}}{\end{small}\end{adjustwidth}\vspace{1em}}
\newenvironment{mykeywords}{\vspace{1em}\begin{adjustwidth}{3em}{3em}\begin{small}\textbf{Keywords.}}{\end{small}\end{adjustwidth}\vspace{1em}}

\DeclareCaptionLabelSeparator{custom}{.}
\DeclareCaptionLabelFormat{custom}
{%
  \textsc{#1 #2}
}
\DeclareCaptionFormat{custom}
{%
  #1#2 #3
}
\captionsetup
{
  format=custom,%
  labelformat=custom,%
  labelsep=custom
}

\begin{document}

\begin{center}
  {\Large\bfseries Math 342 Tutorial} \\
  {\normalsize\bf May 21, 2025}
\end{center}

\noindent{\bf Question 1.} Prove every integer $n$ with $\abs{n}>1$ is either
prime or can be factored into a product of prime numbers [Hint: use the
principle of strong mathematical induction]. \\

\blue{We use induction, and we note it suffices to show the result for $n>0$.
Certainly, the result is true for $n=2$. Next let $n>2$. If $n$ is prime, we're
done. If $n$ is composite, then there exists $a,\,b \neq 1$ such that $n=ab$.
But $1<a,\,b < n$ so that $a$ and $b$ are either prime or a product of prime
numbers. But then $n=ab$ is a product of prime numbers. By the principle of
strong mathematical induction, the result follows.} \\

\noindent{\bf Question 2.} Use Question 1 to show there are infinitely many
prime numbers [Hint. use contradiction and consider the number $N=1+p_1 \cdots
p_n$ where $p_1,\dots,\,p_n$ are the assumed finite number of primes]. \\

\blue{Towards a contradiction, assume there are only a finite number of primes,
  say $p_1,\dots,\,p_n$. Define $N=1+p_1 \cdots p_n$. From Question 1, either
  $N$ is a prime or a product of primes. But $N>p_i$ for all $i \in
  \{1,\dots,\,n\}$, hence $N$ cannot be prime. Therefore, $N$ is a product of
  the primes $p_1,\dots,\,p_n$. Observe, however, that no prime
  $p_1,\dots,\,p_n$ can divide $N$ for if some $p_i \mid N$, then $p_i \mid 1$
  which cannot be. So $N$ cannot be a product of the primes
  $p_1,\dots,\,p_n$, a contradiction. This establishes the result.} \\

\noindent{\bf Question 3.} The gcd of a multiset $\{a_1,\dots,\,a_n\}$ of
integers is defined inductively by
$(a_1,\,a_2,\dots,\,a_n)=(a_1,\,(a_2,\dots,\,a_n))$. Show
\begin{enumerate*}[{\bf (a)}]
\item the gcd of $\{a_1,\dots,\,a_n\}$ is independent of the ordering chosen for
  the elements of the set, and
\item there exists integers $x_1,\dots,\,x_n$ such that $(a_1,\,a_2,\dots,\,a_n)
  = x_1a_1 + \cdots + x_na_n$.
\end{enumerate*}

\blue{
  \begin{enumerate}[(a)]
  \item We show the following by induction:
    \begin{enumerate}[(1)]
    \item $(a_1,\dots,\,a_n) \mid a_i$ for each $1 \leqq i \leqq n$, and
    \item if any $d \mid a_i$ for each $1 \leqq i \leqq n$, we have $d \mid
      (a_1,\dots,\,a_n)$.
    \end{enumerate}
    In particular, we start by showing that $(a_1,\dots,\,a_n)$ is the largest
    common divisor of $a_1,\dots,\,a_n$.
    The base case ($n=2$) was shown in class. Let $n > 2$, and assume the result
    for $n-1$, that is, the result holds for every multisubset of $\Z$ of
    cardinality $n-1$. Let $\{b_1,\dots,b_n\}$ be an arbitrary multisubset of
    $\Z$. Since
    \[
      (b_1,\dots,\,b_n) = (b_1,\,(b_2,\dots,\,b_n)),
    \]
    we have that $(b_1,\dots,\,b_n)$ divides both $b_1$ and $(b_2,\dots,\,b_n)$.
    Since $(b_2,\dots,\,b_n)$ divides each of $b_2,\dots,\,b_n$, so does
    $(b_1,\dots,\,b_n)$. This shows (1) holds for $\{b_1,\dots,\,b_n\}$. Next,
    let $d$ be such that $d \mid a_i$ for each $1 \leqq i \leqq n$. Since then
    $d \mid b_i$ for $i \geqq 2$, we have that $d \mid (b_2,\dots,\,b_n)$ by the
    inductive hypothesis. Since $d \mid b_1$ and $d \mid (b_2,\dots,\,b_n)$, we
    have that $d \mid (b_1,\,(b_2,\dots,\,b_n))=(b_1,\dots,\,b_n)$. This shows
    that (2) holds for $\{b_1,\dots,\,b_n\}$. Since $\{b_1,\dots,\,b_n\}$ was an
    arbitrary multisubset of $\Z$ of cardinality $n$, it holds for every
    multisubset of cardinality $n$. It now follow that (1) and (2) hold for
    every multisubset of $\Z$ of finite order. \\
    Let $\{a_1,\dots,\,a_n\}$ be an arbitrary multisubset of $\Z$, and let
    $\sigma$ be an arbitrary permutation of $\{a_1,\dots,\,a_n\}$. From (1) and
    (2) above, we have that both $(a_1,\dots,\,a_n) \mid
    (\sigma(a_1),\dots,\,\sigma(a_n))$ and $(\sigma(a_1),\dots,\,\sigma(a_n))
    \mid (a_1,\dots,\,a_n)$. Therefore,
    $(a_1,\dots,\,a_n)=(\sigma(a_1),\dots,\,\sigma(a_n))$ as these are both
    positive values by assumption. Since $\{a_1,\dots,\,a_n\}$ and $\sigma$ were
    arbitrary, the result follows.
  \item Given a multisubset
$\{a_1,\dots,\,a_n\}$ of $\Z$, we assume the indexing is chosen so that $a_i
\leqq a_j$ whenever $i < j$. Note the logical form of the proposition we are
required to prove. We must show for all $n>1$, the result holds for every
multisubset of $\Z$ of cardinality $n$. We have shown the base case---that is,
$n=2$---in class already. Let $n>2$, and suppose the result holds for $n-1$;
namely, for every multisubset $\{a_1,\dots,\,a_{n-1}\}$ of $\Z$ of cardinality
$n-1$, there exists integers $x_1,\dots,\,x_{n-1}$ such that
$a_1x_1+\cdots+a_{n-1}x_{n-1}=(a_1,\dots,\,a_{n-1})$. Let $\{b_1,\dots,\,b_n\}$
be an arbitrary multisubset of $\Z$ of cardinality $n$. From the base case,
there exists integers $y_1,\,y_2$ such that
\[
  (b_1,\dots,\,b_n) = (b_1,\,(b_2,\dots,\,b_n)) = y_1b_1 + y_2(b_2,\dots,\,b_n).
\]
From the inductive hypothesis, there exists integers $z_2,\dots,\,z_n$ such that
\[
  (b_2,\dots,\,b_n) = z_2b_2+\cdots+z_nb_n.
\]
Putting things together, we have
\begin{align*}
  (b_1,\dots,\,b_n) &= (b_1,\,(b_2,\dots,\,b_n)) \\
                    &= y_1b_1+y_2(b_2,\dots,\,b_n) \\
                    &= y_1b_1+y_2z_2b_2+\cdots+y_2z_nb_n.
\end{align*}
Since $\{b_1,\dots,\,b_n\}$ was chosen arbitrarily, the result holds for $n$ as
well. The result now follows from the principle of mathematical induction.
\end{enumerate}}

\noindent{\bf Question 4.} Let $p$ be a prime. Show that if $p \mid ab$ then $p
\mid a$ or $p \mid b$. \\

\blue{Suppose that $p$ divides neither $a$ nor $b$ but $p \mid ab$. In
particular, $(a,\,p)=(b,\,p)=1$. Hence, there exists integers $w$, $x$, $y$, and
$z$ such that
  \[
    1 = wp+xa = yp+zb.
  \]
  Then
  \[
    1=(wp+xa)(yp+zb) = wyp^2+(wzb+xay)p+xzab.
  \]
  Since $p \mid ab$ by assumption, we have that $p \mid
  wyp^2+(wzb+xay)p+xzab=1$, a contradiction. Therefore, $p$ does not divide
  $ab$.} \\

\noindent{\bf Question 5.} If $(a,\,b)=1$, then $(a+b,\,a-b)=1\text{ or }2$. \\

\blue{Let $d=(a+b,a-b)$. We have that $d \mid a+b$ and $d \mid a-b$. Therefore,
  $d \mid (a+b)+(a-b)=2a$ and $d \mid (a+b)-(a-b)=2b$. By assumption, there
  exists integers $x,\,y$ such that $ax+by=1$ so that $2ax+2bx=2$. Since $d \mid
  2a,\,2b$, it must be that $d \mid 2ax+2bx=2$. Therefore, $d=1\text{ or }2$ (up
  to negation).} \\

\noindent{\bf Question 6.} If $(a,\,b)=1$, then $(a+b,\,a^2-ab+b^2)=1\text{ or
}3$. \\

\blue{Let $d=(a+b,\,a^2-ab+b^2)$, and note that
$d=(a+b,\,a^2-ab+b^2)=(a+b,\,(a+b)^2-3ab)=(a+b,\,3ab)$. Suppose that $d>1$ and
$p$ is a prime divisor of $d$. Then $p \mid a+b$, and either $p \mid 3$ or $p
\mid a$ or $p \mid b$. If $p \mid a$, then $p \mid (a+b)-a=b$ contradicting the
fact that $a$ and $b$ are coprime. Thus, $p$ does not divide $a$. Similarly, $p$
does not divide $b$. Hence, $p$ divides 3. But 3 is prime so that $p=3$. We
have, therefore, that $d=1 \text{ or }3^k$ for some $k>0$. Suppose next that
$k>1$. The single prime divisor 3 of $d=3^k$ must divide $3ab$. But we have
already seen that 3 cannot divide $a$ or $b$, so it must (trivially) divide the
remaining factor 3. Inductively, $d=3^k \mid 3$, contradicting the the
assumption that $k>1$. We finally have that $d=1\text{ or }3$.} \\

\noindent{\bf Question 7.} If $(a,\,b)=1$, then $(a^n,\,b^k)=1$ for all $n,\,k
\geqq 1$. \\

\blue{Recall that $(a,\,b)=1$ if and only if $a$ and $b$ have no common factors
that are not 1. Let $n,\,k > 1$ be given, and suppose that $(a^n,\,b^k)=d$ with
$d > 1$. Let $p$ be a prime divisor of $d$. Then $p \mid a^n,\,b^k$; in
particular, $p \mid a,\,b$, a contradiction. Therefore, $d=1$.} \\

\noindent{\bf Question 8.} If $2^n-1$ is a prime, then $n$ is a prime. \\

\blue{Towards a contradiction, suppose that $n$ is composite, say $n=xy$ with
  $1<x,\,y<n$. Then
  \[
    2^{xy}-1 = (2^x-1)(2^{x(y-1)}+2^{x(y-2)}+\cdots+2^x+1).
  \]
  However, $2^x-1 > 1$ since $x>1$, hence $2^x-1$ is a nontrivial divisor of
  $2^n-1$. This contradicts the assumption that $2^n-1$ is prime; thus, it must be
  that $n$ is prime.} \\

\noindent{\bf Question 9.} If $2^n+1$ is a prime, then $n$ is a power of 2. \\

\blue{Towards a contradiction, suppose that $n=2^km$ with $m>1$ odd. Then
  \[
    2^{2^km}+1=(2^{2^k})^m+1 = (2^{2^k}+1)(2^{2^k(m-1)}-2^{2^k(m-2)}+\cdots-2^{2^k}+1).
  \]
  But $2^{2^k}+1 \geqq 2$. Since $m>1$, we have that $2^{2^km}+1 > 2^{2^k}+1$
and $2^{2^k(m-1)}-2^{2^k(m-2)}+\cdots-2^{2^k}+1 > 1$. We have exhibited
factorization of $2^n+1$ into a product of two nontrivial divisors contradicting
the assumption that $2^n+1$ is prime. Therefore, $m=1$ and $n$ is a power of 2.}
\\

\noindent{\bf Question 10.}
\begin{enumerate*}[{\bf (a)}]
\item Suppose that $(a,\,b) = (c,\,d) = 1$ and $\frac{a}{b}+\frac{c}{d}=n$ is an
  integer. Show $\abs{b}=\abs{d}$.
\item Prove the sum $\sum_{k=1}^n\frac{1}{k}$ is not an integer for $n>1$ [Hint.
  show the sum can be written as $\frac{a}{b}$ with $a$ and $b$ of opposite
  parity].
\end{enumerate*} \\

\blue{
  \begin{enumerate}[{\bf (a)}]
  \item We have that
    \[
      \frac{a}{b}+\frac{c}{d} = \frac{ad+bc}{bd} = n.
    \]
    In particular, we have that $bd \mid ad+bc \Rightarrow b \mid ad+bc
    \Rightarrow b \mid ad$. Since $(a,\,b)=1$, it follows that $b \mid d$.
    Similarly, $d \mid b$. Thus, $\abs{b}=\abs{d}$.
  \item We claim the sum always evaluates to a fraction $\frac{a}{b}$ with $a$
    odd and $b$ even. The proof is by induction on $n$. For $n=2$, we have
    $1+\frac{1}{2} = \frac{3}{2}$, so the base case holds. Let $n>2$ be given,
    and suppose the result holds for all $m$ with $m < n$ and $m < 2$.
    Partitioning the sum $\sum_{k=1}^n\frac{1}{k}$ into whether the denominator
    is even or odd, we find that
    \[
      \sum_{k=1}^n\frac{1}{k} = 1+\cdots+\frac{1}{k} = \sum_{k=2}^{\lceil n/2
        \rceil}\frac{1}{2k-1} + \frac{1}{2}\sum_{k=1}^{\lfloor n/2
        \rfloor}\frac{1}{k}.
    \]
    By our inductive hypothesis, we can write $\frac{1}{2}\sum_{k=1}^{\lfloor
      n/2 \rfloor}\frac{1}{k}$ as $\frac{a}{b}$ with $a$ odd and $b$ even. Next,
    observe that $\sum_{k=2}^{\lceil n/2
      \rceil}\frac{1}{2k-1}=\frac{f(k)}{(2k-1)!!}$ where $f(k)$ is a polynomial
    in $k$. Since $(2k-1)!!$ is odd, it follows that $\sum_{k=2}^{\lceil n/2
      \rceil}\frac{1}{2k-1}=\frac{f(k)}{(2k-1)!!}$ can be written as
    $\frac{c}{d}$ with $d$ odd. Then
    \[
      \sum_{k=1}^n\frac{1}{k} = \frac{c}{d}+\frac{a}{b} = \frac{cb+ad}{db}.
    \]
    Since $a$ and $d$ are odd, $ad$ is odd. Also, $cb$ is even as $b$ is even.
    Therefore, $cb+ad$ is odd. Finally, $db$ is even as $b$ is even. Thus, the
    result holds for $n$ as well and the proposition follows by the strong
    principle of mathematical induction.
\end{enumerate}}

\noindent{\bf Question 11.} Prove:
\begin{enumerate*}[{\bf (a)}]
\item For every integer $k$ the numbers $2k+1$ and $9k+4$ are relatively prime.
\item For every integer $k$, express the gcd of $2k-1$ and $9k+4$ as a function
  of $k$.
\end{enumerate*} \\

\blue{
  \begin{enumerate}[{\bf (a)}]
  \item We solve $x(2k+1)+y(9k+4)=1$. This leads to the system $2x+9y=0,\,x+4y$
    of linear equations. This system has the unique solution $x=9$ and $y=-2$.
    Therefore $9(2k+1)-2(9k+4)=1$, which shows $(2k+1,\,9k+4)=1$.
  \item We have
    \[
      (2k-1,\,9k+4) = (2k-1,\,4(2k-1)+(k+8)) = (k+8,\,2k-1) = (k+8,\,2(2k-1)-17)
      = (k+8,\,17).
    \]
    So $(2k-1,\,9k+4)=1\text{ or }17$. But $(2k-1,\,9k+4)=17$ if and only if
    $k=17m+9$ for some integer $m$. In all other cases, $(2k-1,\,9k+4)=1$.
  \end{enumerate}
}

\noindent{\bf Question 12.} Prove that for positive integers $m$ and $a$ we have
\[
  \left(
    \frac{a^m-1}{a-1},\,a-1
  \right)
  = (a-1,\,m).
\]

\blue{Let $d=(\frac{a^m-1}{a-1},\,a-1)$, and observe
  \begin{align*}
    \frac{a^m-1}{a-1} &= a^{m-1} + a^{m-2} + \cdots + a + 1 \\
                      &= (a^{m-1}-1)+(a^{m-2}-1)+\cdots+(a-1)+m.
  \end{align*}
  Since $a-1 \mid a^k-1$ for $k \geqq 0$, we have that $d \mid m$ whereupon $d
  \mid (a-1,\,m)$. Conversely, since $(a,\,m) \mid a-1,\,m$, it follows that
  $(a,\,m) \mid \frac{a^m-1}{a-1}$ and hence $(a,\,m) \mid
  (\frac{a^m-1}{a-1},\,a-1)=d$. Thus $d=(a-1,\,m)$.}

\end{document}

