\documentclass[a4paper,11pt]{article}

\usepackage[utf8]{inputenc}
\usepackage[english]{babel}
\usepackage{amssymb, amsmath, amsthm, mathrsfs}
\usepackage[left=1.0in,right=1.0in,top=1.0in,bottom=1.0in]{geometry}
\usepackage[T1]{fontenc}
\usepackage{array}
\usepackage{longtable}
\usepackage{multirow}
\usepackage{calc}
\usepackage[inline,shortlabels]{enumitem}
\usepackage{changepage}
\usepackage{booktabs}
\usepackage{capt-of}
\usepackage{subcaption}
\usepackage[leftcaption]{sidecap}
\usepackage[numbers]{natbib}
\usepackage{times}
\usepackage{titlesec}
\usepackage{xcolor}
\usepackage{lineno}
\usepackage{xpatch}
\xpatchcmd\swappedhead{~}{.~}{}{}
\allowdisplaybreaks

\newtheoremstyle{mythm}
{}                % Space above
{}                % Space below
{\itshape}        % Theorem body font % (default is "\upshape")
{1.5em}                % Indent amount
{\scshape}       % Theorem head font % (default is \mdseries)
{.}               % Punctuation after theorem head % default: no punctuation
{0.5em}               % Space after theorem head
{}                % Theorem head spec
\theoremstyle{mythm}


\newtheorem*{theorem*}{Theorem}
\newtheorem{theorem}{Theorem}
\newtheorem{fact}[theorem]{Fact}
\newtheorem{proposition}[theorem]{Proposition}
\newtheorem{lemma}[theorem]{Lemma}
\newtheorem{corollary}[theorem]{Corollary}
\newtheorem{question}[theorem]{Question}
\newtheorem{result}[theorem]{Result}
\newtheorem{observation}[theorem]{Observation}
\newtheorem{conjecture}[theorem]{Conjecture}

\newtheoremstyle{mydef}
{}                % Space above
{}                % Space below
{}        % Theorem body font % (default is "\upshape")
{1.5em}                % Indent amount
{\scshape}       % Theorem head font % (default is \mdseries)
{.}               % Punctuation after theorem head % default: no punctuation
{0.5em}               % Space after theorem head
{}                % Theorem head spec
\theoremstyle{mydef}

\newtheorem{example}[theorem]{Example}
\newtheorem{definition}[theorem]{Definition}
\newtheorem{remark}[theorem]{Remark}
\newtheorem*{remark*}{Remark}

\makeatletter
\renewenvironment{proof}[1][\proofname]{\par
  \pushQED{\qed}%
  \normalfont \topsep6\p@\@plus6\p@\relax
  \trivlist
\item\relax
  {\hspace{1.5em}\itshape
    #1\@addpunct{.}}\hspace\labelsep\ignorespaces
}{%
  \popQED\endtrivlist\@endpefalse
}
\makeatother

\def\Box{\hskip1ex\vbox{\hrule height0.6pt\hbox{%
      \vrule height1.3ex width0.6pt\hskip0.8ex
      \vrule width0.6pt}\hrule height0.6pt
  }}
\renewcommand{\qed}{\Box}

\newcommand{\red}[1]{\textcolor{red}{#1}}
\newcommand{\blue}[1]{\textcolor{blue}{#1}}
\newcommand{\purple}[1]{\textcolor{magenta}{#1}}
\newcommand{\ddet}{\text{det}}
\renewcommand{\pmod}[1]{\text{ (mod $#1$)}}
\newcommand{\mmod}[2]{#1\text{ mod }#2}
\newcommand{\abs}[1]{\left\vert #1 \right\vert}
\newcommand{\C}{\mathbf{C}}
\newcommand{\Z}{\mathbf{Z}}
\newcommand{\N}{\mathbf{N}}
\newcommand{\LL}{\mathscr{G}}
\newcommand{\z}{\mathbin{\ooalign{$\hidewidth i \hidewidth$\cr$\phantom{+}$}}}
\newcommand{\y}{\mathbin{\ooalign{$\hidewidth j \hidewidth$\cr$\phantom{+}$}}}
\newcommand{\gf}{\text{GF}}

\newcolumntype{R}{>{\scriptsize}r}
\newcolumntype{L}{>{\scriptsize}l}
\newcolumntype{C}{>{\scriptsize}c}

\renewcommand{\citenumfont}[1]{\textbf{#1}}
\renewcommand{\bibnumfmt}[1]{\textbf{#1.}}

\titleformat{\section}{\normalfont\Large\bfseries\centering}{\thesection.}{0.5em}{}
\titleformat{\subsection}{\normalfont\bfseries}{\thesubsection.}{0.5em}{}

\newenvironment{myabstract}{\vspace{1em}\begin{adjustwidth}{3em}{3em}\begin{small}\textbf{Abstract.}}{\end{small}\end{adjustwidth}\vspace{1em}}
\newenvironment{mykeywords}{\vspace{1em}\begin{adjustwidth}{3em}{3em}\begin{small}\textbf{Keywords.}}{\end{small}\end{adjustwidth}\vspace{1em}}

\DeclareCaptionLabelSeparator{custom}{.}
\DeclareCaptionLabelFormat{custom}
{%
  \textsc{#1 #2}
}
\DeclareCaptionFormat{custom}
{%
  #1#2 #3
}
\captionsetup
{
  format=custom,%
  labelformat=custom,%
  labelsep=custom
}

\begin{document}

\begin{center}
  {\Large\bfseries Math 342 Tutorial} \\
  {\normalsize\bf June 11, 2025}
\end{center}

\noindent{\bf Question 1.} Find all solutions to following systems of
congruences in two ways: first, using the Chinese Remainder Theorem; and second,
by iteratively solving and substituting linear congruences.

\begin{enumerate}[(a)]
\item $x \equiv 1 \pmod{2},\, x \equiv 2 \pmod{3},\, x \equiv 3 \pmod{5}$.
\item $x \equiv 0 \pmod{2},\, x \equiv 0 \pmod{3},\, x \equiv 1 \pmod{5},\, x
  \equiv 6 \pmod{7}$.
\end{enumerate}

\noindent{\bf Question 2.} Give the following generalization of the Chinese
Remainder Theorem. Let $m_1,\dots,\,m_r$ be pairwise coprime integers, and let
$a_1,\dots,\,a_r$ be given integers such that each $(a_i,\,m_i)=1$. Then the
system $a_1x \equiv b_1 \pmod{m_1}, \dots,\, a_rx \equiv b_r \pmod{m_r}$ has
exactly one solution modulo $m_1 \cdots m_r$. \\

\noindent{\bf Question 3.}
\begin{enumerate*}[(a)]
\item Show that the system of congruences $x \equiv a_1 \pmod{m_1}, \dots,\, x
  \equiv a_r \pmod{m_r}$ has a solution if and only if $(m_i,\,m_j) \mid
  (a_i-a_j)$ for all $i<j$. Show that if a solution exists, then it is unique
  modulo $[m_1,\dots,\,m_r]$. [Hint: succesively substitute linear equations.]
\item Solve the system $x \equiv 4 \pmod{6},\, x \equiv 13 \pmod{15}$.
\item Solve the system $x \equiv 5 \pmod{6},\, x \equiv 3 \pmod{10},\, x \equiv
  8 \pmod{15}$.
\item Does the system $x \equiv 1 \pmod{8},\, x \equiv 3 \pmod{9}$, $x \equiv 2
  \pmod{12}$ have any solutions?
\end{enumerate*} \\

\noindent{\bf Question 4.} Show there are arbitrarily long strings of
consecutive integers each divisible by a perfect square greater than 1. [Hint:
Use CRT to show there is a simultaneous solution to the system $x \equiv 0
\pmod{4}$, $x \equiv -1 \pmod{9}$, $x \equiv -2 \pmod{25}$, $\dots$, $x \equiv
-k+1 \pmod{p_k^2}$ where $p_k$ is the $k$th prime.] \\

\noindent{\bf Question 5.} Let $m=2^{e_0}p_1^{e_1} \cdots p_k^{e_k}$. Show the
congruence $x^2 \equiv 1 \pmod{m}$ has exactly $r^{r+s}$ solutions where $s=0$
is $e_0=0\text{ or }1$, $s=1$ if $e_0=2$, and $s=2$ if $e_0>2$. [Hint: Use
question 12 from the May 28th tutorial set.] \\

\noindent{\bf Question 6.} Find all solutions to the following congruences.
\begin{enumerate*}[(a)]
\item $x^3+8x^2-x-1 \equiv 0 \pmod{121}$.
\item $x^2+4x+2 \equiv 0 \pmod{343}$.
\item $13x^7-42x-649 \equiv 0 \pmod{1323}$.
\end{enumerate*} \\

\noindent{\bf Question 7.} Suppose $(a,\,p)=1$. Use Hensel's Lemma to find a
recursive formula for the solutions of $ax \equiv 1 \pmod{p^k}$ for all positive
integers $k$. \\

\end{document}

