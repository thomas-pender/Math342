\documentclass[a4paper,11pt]{article}

\usepackage[utf8]{inputenc}
\usepackage[english]{babel}
\usepackage{amssymb, amsmath, amsthm, mathrsfs}
\usepackage[left=1.0in,right=1.0in,top=1.0in,bottom=1.0in]{geometry}
\usepackage[T1]{fontenc}
\usepackage{array}
\usepackage{longtable}
\usepackage{multirow}
\usepackage{calc}
\usepackage[inline,shortlabels]{enumitem}
\usepackage{changepage}
\usepackage{booktabs}
\usepackage{capt-of}
\usepackage{subcaption}
\usepackage[leftcaption]{sidecap}
\usepackage[numbers]{natbib}
\usepackage{times}
\usepackage{titlesec}
\usepackage{xcolor}
\usepackage{lineno}
\usepackage{xpatch}
\xpatchcmd\swappedhead{~}{.~}{}{}
\allowdisplaybreaks

\newtheoremstyle{mythm}
{}                % Space above
{}                % Space below
{\itshape}        % Theorem body font % (default is "\upshape")
{1.5em}                % Indent amount
{\scshape}       % Theorem head font % (default is \mdseries)
{.}               % Punctuation after theorem head % default: no punctuation
{0.5em}               % Space after theorem head
{}                % Theorem head spec
\theoremstyle{mythm}


\newtheorem*{theorem*}{Theorem}
\newtheorem{theorem}{Theorem}
\newtheorem{fact}[theorem]{Fact}
\newtheorem{proposition}[theorem]{Proposition}
\newtheorem{lemma}[theorem]{Lemma}
\newtheorem{corollary}[theorem]{Corollary}
\newtheorem{question}[theorem]{Question}
\newtheorem{result}[theorem]{Result}
\newtheorem{observation}[theorem]{Observation}
\newtheorem{conjecture}[theorem]{Conjecture}

\newtheoremstyle{mydef}
{}                % Space above
{}                % Space below
{}        % Theorem body font % (default is "\upshape")
{1.5em}                % Indent amount
{\scshape}       % Theorem head font % (default is \mdseries)
{.}               % Punctuation after theorem head % default: no punctuation
{0.5em}               % Space after theorem head
{}                % Theorem head spec
\theoremstyle{mydef}

\newtheorem{example}[theorem]{Example}
\newtheorem{definition}[theorem]{Definition}
\newtheorem{remark}[theorem]{Remark}
\newtheorem*{remark*}{Remark}

\makeatletter
\renewenvironment{proof}[1][\proofname]{\par
  \pushQED{\qed}%
  \normalfont \topsep6\p@\@plus6\p@\relax
  \trivlist
\item\relax
  {\hspace{1.5em}\itshape
    #1\@addpunct{.}}\hspace\labelsep\ignorespaces
}{%
  \popQED\endtrivlist\@endpefalse
}
\makeatother

\def\Box{\hskip1ex\vbox{\hrule height0.6pt\hbox{%
      \vrule height1.3ex width0.6pt\hskip0.8ex
      \vrule width0.6pt}\hrule height0.6pt
  }}
\renewcommand{\qed}{\Box}

\newcommand{\red}[1]{\textcolor{red}{#1}}
\newcommand{\blue}[1]{\textcolor{blue}{#1}}
\newcommand{\purple}[1]{\textcolor{magenta}{#1}}
\newcommand{\ddet}{\text{det}}
\renewcommand{\pmod}[1]{\text{ (mod $#1$)}}
\newcommand{\mmod}[2]{#1\text{ mod }#2}
\newcommand{\abs}[1]{\left\vert #1 \right\vert}
\newcommand{\C}{\mathbb{C}}
\newcommand{\Z}{\mathbb{Z}}
\newcommand{\LL}{\mathscr{G}}
\newcommand{\z}{\mathbin{\ooalign{$\hidewidth i \hidewidth$\cr$\phantom{+}$}}}
\newcommand{\y}{\mathbin{\ooalign{$\hidewidth j \hidewidth$\cr$\phantom{+}$}}}
\newcommand{\gf}{\text{GF}}

\newcolumntype{R}{>{\scriptsize}r}
\newcolumntype{L}{>{\scriptsize}l}
\newcolumntype{C}{>{\scriptsize}c}

\renewcommand{\citenumfont}[1]{\textbf{#1}}
\renewcommand{\bibnumfmt}[1]{\textbf{#1.}}

\titleformat{\section}{\normalfont\Large\bfseries\centering}{\thesection.}{0.5em}{}
\titleformat{\subsection}{\normalfont\bfseries}{\thesubsection.}{0.5em}{}

\newenvironment{myabstract}{\vspace{1em}\begin{adjustwidth}{3em}{3em}\begin{small}\textbf{Abstract.}}{\end{small}\end{adjustwidth}\vspace{1em}}
\newenvironment{mykeywords}{\vspace{1em}\begin{adjustwidth}{3em}{3em}\begin{small}\textbf{Keywords.}}{\end{small}\end{adjustwidth}\vspace{1em}}

\DeclareCaptionLabelSeparator{custom}{.}
\DeclareCaptionLabelFormat{custom}
{%
  \textsc{#1 #2}
}
\DeclareCaptionFormat{custom}
{%
  #1#2 #3
}
\captionsetup
{
  format=custom,%
  labelformat=custom,%
  labelsep=custom
}

\begin{document}

\begin{center}
  {\Large\bfseries Math 342 Tutorial} \\
  {\normalsize\bf May 21, 2025}
\end{center}

\noindent{\bf Question 1.} Prove every integer $n$ with $\abs{n}>1$ is either
prime or can be factored into a product of prime numbers [Hint: use the
principle of strong mathematical induction]. \\

\noindent{\bf Question 2.} Use Question 1 to show there are infinitely many
prime numbers [Hint. use contradiction and consider the number $N=1+p_1 \cdots
p_n$ where $p_1,\dots,\,p_n$ are the assumed finite number of primes]. \\

\noindent{\bf Question 3.} The gcd of a multiset $\{a_1,\dots,\,a_n\}$ of
integers is defined inductively by
$(a_1,\,a_2,\dots,\,a_n)=(a_1,\,(a_2,\dots,\,a_n))$. Show
\begin{enumerate*}[{\bf (a)}]
\item the gcd of $\{a_1,\dots,\,a_n\}$ is independent of the ordering chosen for
  the elements of the set, and
\item there exists integers $x_1,\dots,\,x_n$ such that $(a_1,\,a_2,\dots,\,a_n)
  = x_1a_1 + \cdots + x_na_n$.
\end{enumerate*} \\

\noindent{\bf Question 4.} Let $p$ be a prime. Show that if $p \mid ab$ then $p
\mid a$ or $p \mid b$. \\

\noindent{\bf Question 5.} If $(a,\,b)=1$, then $(a+b,\,a-b)=1\text{ or }2$. \\

\noindent{\bf Question 6.} If $(a,\,b)=1$, then $(a+b,\,a^2-ab+b^2)=1\text{ or
}3$. \\

\noindent{\bf Question 7.} If $(a,\,b)=1$, then $(a^n,\,b^k)=1$ for all $n,\,k
\geqq 1$. \\

\noindent{\bf Question 8.} If $2^n-1$ is a prime, then $n$ is a prime. \\

\noindent{\bf Question 9.} If $2^n+1$ is a prime, then $n$ is a power of 2. \\

\noindent{\bf Question 10.}
\begin{enumerate*}[{\bf (a)}]
\item Suppose that $(a,\,b) = (c,\,d) = 1$ and $\frac{a}{b}+\frac{c}{d}=n$ is an
  integer. Show $\abs{b}=\abs{d}$.
\item Prove the sum $\sum_{k=1}^n\frac{1}{k}$ is not an integer for $n > 1$.
\end{enumerate*} \\

\noindent{\bf Question 11.} Prove:
\begin{enumerate*}[{\bf (a)}]
\item For every integer $k$ the numbers $2k+1$ and $9k+4$ are relatively prime.
\item For every integer $k$, express the gcd of $2k-1$ and $9k+4$ as a function
  of $k$.
\end{enumerate*} \\

\noindent{\bf Question 12.} Prove that for positive integers $m$ and $a$ we have
\[
  \left(
    \frac{a^m-1}{a-1},\,a-1
  \right)
  = (a-1,\,m).
\]

\end{document}

